\chapter{Einleitung (5\%) }

\section{Motivation}


Kontinuierliche Integration ist ein wachsender Bereich in der Welt
des automatischen Testens und des Software-Designs.

Traditionell ist der Vorläufer des Prozesses der sogenannte Build Server.
Dessen einzige Aufgabe bestand darin,
Regelmässig einen Build eines Software-Projektes zu erstellen.
Resultat war hierbei nur eine kurze Information über Erfolg oder Misserfolg,
zusätzlich wurde bei Erfolg das Resultat (i.d.r. ein ausführbares Programm)
zur Verfügung gestellt.

Im Laufe der Zeit sind jedoch neue Aufgaben hinzugekommen,
inzwischen werden auch Testergebnisse verwaltet und Builds sind Konfigurierbar
und werden in mehr als nur einer Konfiguration Ausgeführt.

Neue Technologien in den Bereichen Quellcode Management und führen zu geaenderten



\begin{verbatim}
- kontinuierliche integration und ihre brauchbarkeit


- zitate aus dem buch

\end{verbatim}

\section{Problem}

Die in der Motivation erwähnten Zukunftsvisionen treffen in der Praxis auf einige harte Probleme,
existierende Systeme sind schwer erweiterbar und haben oft schwer zugängliche Datenmodelle.

Es gibt Praktisch keine Standardschnittstelle um mit CI Systemen und ihren zu integrieren/

Die Kombination aus diesen beiden Problemen erzeugt viele Probleme,
welche später im Kapitel \ref{chap:ist-analyse} behandelt werden.
Resultat ist, das die existierenden Systeme im Bezug auf die Zukunftsvisionen,
nicht oder nur schwer anpassbar sind.

Der Mangel an Standardschnittstellen für den Datenzugriff und
das Fehlen von Standardschnittstellen für die Erweiterung 





\begin{verbatim}
- datenbankferne
- praxisprobleme div akt. existierender systeme
  - erweiterbarkeit
  - einfachheit
  - kombination

\end{verbatim}

\section{Zielsetzung}

Um aufzuzeigen, dass die oben genannten Probleme lösbar sind,
soll die Grundlage für ein System geschaffen werden,
dass auch in der Lage ist all diese Probleme zu lösen.

Ziel ist es, ein Grundlegendes Verteiltes System fuer Kontinuierliche integration zu schaffen,
welches auf breit verfügbarer datenbantechnologie basiert,
und dies als Ansatzpunkt zu benutzen,
um das System zu erweitern und Standardschittstellen fuer den Datenzugriff zur Verfuegung zu stellen.

Es sollte die grundlegenden Funktionen zur Verfügung stellen um alle Operationen
eines grundlegenden CI zu erfuellen,
und die Moeglichkeit geben erweiterungen zu Erstellen'

Beispiele fuer solche Erweiterungen wären wie schon genannt
z.b. vergleichende Analyse zwischen Entwicklungzweigen oder Eigenständige Analysewerkzeuge.

\begin{verbatim}
- loesungswunsch
    - offene fragen
- dokumentorientert und verteilt
\end{verbatim}

\section{Abgrenzung}

\begin{verbatim}
- kernkomponenten
- prototyp
- ausgewaehlte erweiterung

- keine authentifikation
- keine user-interfaces

\end{verbatim}

\section{Aufbau der Arbeit}


