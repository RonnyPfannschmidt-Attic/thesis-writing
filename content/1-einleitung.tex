\chapter{Einleitung (5\%) }

\section{Motivation}

Kontinuierliche Integration ist ein wachsender Bereich
im Gebiet der Software-entwicklung und ihrere Qualit\"atssicherung.

% referenzen folwer/xp
Urspr\"unglich entstammend aus der Praxis der Extreme Programming
\cite{xp:explained} \cite{folwer:xp}
besch\"aftigen sich Werkzeuge f\"ur Kontinuierliche Integration damit,
Programme nach einer vollzogenen Integration auch zu Testen.

In regelm\"assigen Abst\"anden, wie z.b. nach einer erfolgreichen Integration oder nach Arbeitsschluss,
wird das Programm vobereitet und getestet.
Dies unterst\"utzt das rasche Auffinden von Fehlern
und gibt gundlegende Informationen \"uber den Zustand der Software.
Nach Abschluss eines solchen Testes stehen f\"ur gew\"ohnlich das Resultat
und ein Statusbericht zur Verf\"ugung.

Im Laufe der Entwicklung hat sich der Inhalt der Statusberichte weiterentwickelt,
waren es anf\"anglich nur sp\"arliche Daten \"uber Erfolg/Misserfolg,
so findet man Heutzutage Mitschnitte verschiedener Datenquellen, Testergebnisse
\cite{jenkins:junitxml}, sowie detailierte Informationen \"uber die Umgebung des Builds.

Weiterhin entstand mit Zunehmenden Anforderungen an Portabilit\"at \"uber
verschiedendste Arten von Platform (z.B. Datenbanksystem oder Betriebssystem)
die Notwendigkeit das System in mehr als nur einer Umgebung zu Testen.

Auch die Verwaltung von Software hat sich ge\"andert.
W\"ahrend urspr\"unglich nur nur auf einem Entwicklungszweig gearbeitet und/oder integriert wurde,
findet dies Dank moderner SCM in vielen Entwicklungszweigen statt.
\cite{dvcs:vorteile} \cite{dvcs:entwicklungsmodelle}

Aufgrund dieser Entwicklung stehen immer mehr Daten und Informationen zur Verf\"ugung,
und es erscheint \"uberaus sinnvoll, diese auch zu nutzen, um den Entwicklungsprozess zu unterst\"utzen.



\section{Problem}

Die in der Motivation erw\"ahnte Zukunftsvision die Daten zu nutzen,
trifft in der Praxis auf einige harte Probleme.
Existierende Systeme sind schwer erweiterbar und haben oft schwer zugängliche Datenmodelle.

Es gibt keine Standardschnittstelle um mit CI Systemen und ihren Daten zu arbeiten

Die Kombination aus diesen beiden Problemen erzeugt viele Probleme,
welche später im Kapitel \ref{chap:ist-analyse} behandelt werden.

Resultat ist, das die existierenden Systeme im Bezug auf die Zukunftsvisionen,
nicht oder nur schwer anpassbar sind.

Der Mangel an Standardschnittstellen für den Datenzugriff und
das Fehlen von Standardschnittstellen für die Erweiterung der CI Systeme
erschwert die Implementation von Analysewerkzeugen ungemein.





\begin{verbatim}
- datenbankferne
- praxisprobleme div akt. existierender systeme
  - erweiterbarkeit
  - einfachheit
  - kombination


- probleme des client/server ?

\end{verbatim}

\section{Zielsetzung}

Um aufzuzeigen, dass die oben genannten Probleme lösbar sind,
soll die Grundlage für ein System geschaffen werden,
welches auch in der Lage ist diese Probleme zu lösen.

Ziel ist es, ein Grundlegendes Verteiltes System f\"ur Kontinuierliche integration zu schaffen,
welches auf verteilter Datenbantechnologie basiert,
und dies als Ansatzpunkt zu benutzen,
um das System zu erweitern und Standardschittstellen f\"ur den Datenzugriff zur Verf\"ugung zu stellen.

Es sollte die grundlegenden Funktionen zur Verfügung stellen um alle Operationen
eines grundlegenden CI zu erf\"ullen,
und die Moeglichkeit geben Erweiterungen zu Erstellen'

Beispiele fuer solche Erweiterungen wären wie schon genannt
z.b. vergleichende Analyse zwischen Entwicklungzweigen oder Eigenständige Analysewerkzeuge.

\begin{verbatim}
- loesungswunsch
    - offene fragen
- dokumentorientert und verteilt
\end{verbatim}

\section{Abgrenzung}

\begin{verbatim}
- kernkomponenten
- prototyp
- ausgewaehlte erweiterung

- keine authentifikation
- keine user-interfaces

\end{verbatim}

\section{Aufbau der Arbeit}


