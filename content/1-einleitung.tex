\chapter{Einleitung (5\%) }
\label{chap:intro}
\section{Motivation}

Kontinuierliche Integration ist ein wachsender Bereich
in der Welt der Software-Entwicklung und ihrer Qualitätssicherung.

% referenzen folwer/xp
Entstammend aus der Praxis des ``Extreme Programming''
\cite{xp:explained, folwer:xp},
beschäftigen sich Werkzeuge für kontinuierliche Integration damit,
Programme nach einer vollzogenen Integration zu testen.

In regelmäßigen Abständen, wie z.B. nach einer erfolgreichen Integration oder nach Arbeitsschluss,
wird das Programm vorbereitet und getestet.
Dies unterst\"utzt das rasche Auffinden von Fehlern
und informiert \"uber den Zustand der Software.
Nach Abschluss eines solchen Testes stehen dann das Resultat
und ein Zustandsbericht zur Verf\"ugung.

Im Laufe der Zeit hat sich der Inhalt der Zustandsberichte weiterentwickelt.
Waren es anf\"anglich nur sp\"arliche Daten \"uber Erfolg/Misserfolg,
so findet man heutzutage Mitschnitte verschiedener Datenquellen, Testergebnisse
\cite{jenkins:junitxml}, sowie detaillierte Informationen \"uber die Umgebung des Tests.

Weiterhin entstand mit Zunehmenden Anforderungen an Portabilit\"at \"uber
verschiedenste Arten von Plattformen (z.B. Datenbank oder Betriebssysteme)
die Notwendigkeit das System in mehr als nur einer Umgebung zu testen.

Auch die Verwaltung von Software hat sich ge\"andert.
Während ursprünglich nur auf einem Zweig der Entwicklung gearbeitet und integriert wurde,
findet dies Dank moderner Quellcode-verwaltung jetzt in vielen Zweigen statt.
\cite{dvcs:vorteile, dvcs:entwicklungsmodelle}

Aufgrund dieser Entwicklung stehen immer mehr Daten und Informationen zur Verfügung,
und es erscheint überaus sinnvoll, diese auch zu nutzen, um den Entwicklungsprozess zu unterstützen.

Besonders Interessant sind dabei feststellbare Unterschiede zwischen den Umgebungen,
sowie feststellbare Unterschiede zwischen verschiedenen Zweigen der Entwicklung.

%XXX: test vor integration?


\begin{verbatim}
start traditionelles ci
ueberfuehrung anforderungen
ueberfuehrung branches, build achsen
groessere datenmengen
analyse
feststellung struktir
\end{verbatim}



\section{Problem}

Die Zukunftsvision, die Daten zu nutzen,
trifft in der Praxis auf einige harte Probleme.
Existierende Systeme sind schwer erweiterbar und haben oft schwer zugängliche Datenmodelle.

Es gibt keine Standards für Schnittstellen um CI-Systeme zu erweitern und mit ihren Daten zu arbeiten

Die Kombination dieser beiden Problemen erzeugt wiederum viele Probleme,
welche später im Kapitel \ref{chap:ist-analyse} behandelt werden.

Das Resultat ist, dass die existierenden Systeme im Bezug auf die Zukunftsvisionen,
nicht oder nur schwer anzupassen sind.

Der Mangel an Schnittstellen für den Datenzugriff und
das Fehlen von Schnittstellen für die Erweiterung der CI-Systeme
erschwert die Implementation von Werkzeugen für Analyse ungemein.


\begin{verbatim}
- datenbankferne
- probleme des client/server ?
\end{verbatim}

\section{Zielsetzung}
%XXX nochmal
Ziel dieser Arbeit ist es  aufzuzeigen, dass die oben genannten Probleme lösbar sind.
Zu diesem Zweck soll die Grundlage für ein System geschaffen werden,
welches auch in der Lage ist dies umzusetzen.

Es ist somit ein grundlegendes verteiltes System f\"ur Kontinuierliche Integration zu schaffen,
welches auf verteilter Datenbanktechnologie basiert.
Dies dient als Ansatzpunkt, um das System zu erweitern und
Standardschittstellen f\"ur den Datenzugriff zur Verf\"ugung zu stellen.

Dabei sollten sowohl die grundlegenden Funktionen eines CI,
als auch die M\"oglichkeit der Erweiterung in Betracht gezogen werden,

Beispiele f\"ur solche Erweiterungen wären,
wie schon genannt,
vergleichende Analyse zwischen Entwicklungzweigen oder Eigenständige Analysewerkzeuge.

Wichtig ist hierbei die grundlegende Erweiterbarkit
und nicht die L\"osung eines speziellen Problemes.


\begin{verbatim}
- loesungswunsch
    - offene fragen
- dokumentorientert und verteilt
\end{verbatim}

\section{Abgrenzung}

\begin{verbatim}
- kernkomponenten
- prototyp
- ausgewaehlte erweiterung
  - praktisch analyse (sommer)
  - theoretish - regressions-vergleiche

- keine authentifikation
- keine user-interfaces

\end{verbatim}

\section{Aufbau der Arbeit}


