\chapter{Einleitung}
\label{chap:intro}

\section{Motivation}
\label{sec:intro:motiv}

\ac{CI} ist ein wachsender Bereich
in der Welt der Software-Entwicklung und ihrer Qualitätssicherung.

Entstammend aus der Praxis des \ac{XP}
\cite{xp:explained, folwer:xp},
beschäftigen sich Werkzeuge für \ac{CI} damit,
Programme nach einer vollzogenen Integration zu testen.

In regelmäßigen Abständen, wie z.B. nach einer erfolgreichen Integration oder nach Arbeitsschluss,
wird das Programm vorbereitet und getestet.
Dies unterstützt das rasche Auffinden von Fehlern
und informiert über den Zustand der Software.
Nach Abschluss eines solchen Tests stehen dann das Resultat
und ein Zustandsbericht zur Verfügung.

Im Laufe der Zeit hat sich der Inhalt der Zustandsberichte weiterentwickelt.
Waren es anfänglich nur spärliche Informationen über Erfolg bzw. Misserfolg,
so findet man heutzutage Mitschnitte verschiedener Datenquellen, Testergebnisse
\cite{jenkins:junitxml} und detaillierte Informationen über die Umgebung des Tests.

Weiterhin entstand mit zunehmenden Anforderungen an die Portabilität über
verschiedenste Arten von Plattformen (z.B. Datenbanken oder Betriebssystemen)
die Notwendigkeit das System in mehr als nur einer Umgebung zu testen.

Die Verwaltung von Software hat sich geändert.
Während ursprünglich nur auf einem Zweig der Entwicklung gearbeitet und integriert wurde,
findet dies Dank moderner Quellcodeverwaltung jetzt in vielen Zweigen statt.

Aufgrund dieser Entwicklung stehen immer mehr Daten und Informationen zur Verfügung
und es erscheint überaus sinnvoll, diese auch zu nutzen, um den Entwicklungsprozess zu unterstützen.

%XXX woanders?
Besonders interessant sind dabei feststellbare Unterschiede zwischen den Umgebungen,
sowie feststellbare Unterschiede zwischen verschiedenen Zweigen der Entwicklung.

%XXX: test vor integration?


%\begin{verbatim}
%start traditionelles ci
%ueberfuehrung anforderungen
%ueberfuehrung branches, build achsen
%groessere datenmengen
%analyse
%feststellung struktir
%\end{verbatim}



\section{Problem}
\label{sec:intro:prob}

Die Zukunftsvision, die Daten zu nutzen,
trifft in der Praxis auf einige harte Probleme.
Existierende Systeme sind meistens schwer erweiterbar und haben oft schwer zugängliche Datenmodelle.

Es gibt keine Standards für Schnittstellen um Systeme für \ac{CI} zu erweitern und mit ihren Daten zu arbeiten.

Die Kombination von diesen beiden Problemen erzeugt wiederum viele Probleme,
welche später im Kapitel \ref{chap:ist-analyse} behandelt werden.

Das Resultat dieser Probleme ist, dass die existierenden Systeme im Bezug auf die Zukunftsvisionen,
nicht oder nur schwer anzupassen sind.

Der Mangel an Schnittstellen für den Datenzugriff und
das Fehlen von Schnittstellen für die Erweiterung der \ac{CI}-Systeme
erschwert die Implementation von Werkzeugen für die Analyse ungemein.


\section{Zielsetzung}
\label{sec:intro:ziel}
%XXX nochmal
Ziel dieser Arbeit ist es aufzuzeigen, dass die oben genannten Probleme lösbar sind.
Zu diesem Zweck soll der Kern für ein System geschaffen werden,
welches in der Lage ist dies umzusetzen.

Es ist somit ein verteiltes System für \ac{CI} zu schaffen,
welches auf Datenbanktechnik basiert.
Dies dient als Ansatzpunkt, um das System zu erweitern und
Schnittstellen für den Datenzugriff zur Verfügung zu stellen.

Dabei sollten sowohl die grundlegenden Funktionen eines \ac{CI},
als auch die Möglichkeit der Erweiterung in Betracht gezogen werden.
Beispiele für solche Erweiterungen sind, wie schon genannt,
vergleichende Analyse zwischen Entwicklungszweigen oder eigenständige Analysewerkzeuge.

Wichtig ist hierbei, die grundlegende Erweiterbarkeit aufzuzeigen
und nicht die Lösung eines speziellen Problems.

\section{Abgrenzung}
\label{sec:intro:abgrenzung}

Es soll lediglich der \emph{Kern} eines Systems als \emph{Prototyp} geschaffen werden.
Damit sind Punkte wie die Benutzeroberfläche und Benutzerverwaltung
außen vor gestellt.

Um die Arbeit überschaubar zu halten, wird dabei darauf verzichtet, Analysewerkzeuge außerhalb der verwendeten Datenbank zu betrachten.
Stattdessen wird der Kern des Systems und die Datenbank in den Fokus gestellt.


\section{Aufbau der Arbeit}
\label{sec:intro:aufbau}

Nachdem zuerst einige grundlegende Begriffe in \cref{chap:base} geklärt werden, geht dann \cref{chap:ist-analyse} auf den gegenwärtigen Zustand von Kontinuierlicher Integration ein.

Dabei wird in \cref{sec:ist-analyse:eigenschaften} geklärt, wie aktuelle Systeme aufgebaut sind. Anschließend werden in \cref{sec:ist-analyse:probleme} die Probleme dieser Art von System erläutert. Um dies zu untermauern werden diese dann in \cref{sec:ist-analyse:praxis} an ausgewählten Systemen aufgezeigt.
Anschließend wird das Kapitel in \cref{sec:ist-analyse:zusammenfassung} zusammengefasst.

Nachfolgend befasst sich \cref{chap:target} mit den Kriterien und Zielen für die Lösung der aufgezeigten Probleme. Dabei werden zuerst in \cref{sec:target:systemanforderungen} die Anforderungen an das Gesamtsystem gestellt. Darauf folgend werden in \cref{sec:target:usecases} funktionale Anforderungen und Usecases festgestellt. Nachdem dann in \cref{sec:target:tests} die Tests und Testkriterien festgelegt sind, werden in \cref{sec:target:abgrenzung} Abgrenzungen getroffen. Danach wird das Kapitel in \cref{sec:target:zusammenfassung} zusammengefasst.

Nun befasst sich \cref{chap:design} mit der theoretischen Umsetzung.
Dabei gehen \cref{sec:design:sysarch,sec:design:schema} auf die Systemarchitekture und das Datenschema ein. Darauf folgend beschäftigt sich \cref{sec:design:logik} mit der Logik der einzelnen Komponenten. Dies untermauernd werden dann in \cref{sec:design:bes-ansaetze} besondere Ansätze zur Interaktion mit der Datenbank vorgestellt.

Darauf folgend behandeln \cref{sec:schritt-kontext,sec:design:erweiterungen} die Abarbeitung von Arbeitschritten und die beispielhaften Erweiterungen.
Anschließend kann das Kapitel in \cref{sec:design:zusammenfassung} zusammengefasst werden.

Nun behandelt \cref{cha:tech} die technische Konzeption.


Schließlich wird dann in \cref{cha:imp} die Implementation behandelt.
Dabei wird in \cref{sec:imp:tools} auf die verwendeten Werkzeuge und Produkte eingegangen. Dann wird kurz die Projektstruktur (\cref{sec:imp:projektstruktur}) und grundlegende Primitiven (\cref{sec:imp:primitiven}) vorgestellt.
Schließlich werden die in \cref{sec:imp:komponenten} vorgestellten Komponenten in \cref{sec:imp:kombination} zu einem funktionierenden System kombiniert.
Nun werden noch kurz das Abarbeiten von Arbeitsschritten (\cref{sec:imp:procdir}) sowie die Beispielerweiterungen (\cref{sec:imp:extension}) vorgestellt,
dann wird das Kapitel in \cref{sec:imp:zusammenfassung} zusammengefasst.

Darauf folgend behandelt \cref{cha:eval} die Analyse des Prototypen und anschließend werden in \cref{cha:opt} einige aufgetretene Probleme vorgestellt und beseitigt.

Abschließend wird in \cref{cha:fazit} ein Fazit gezogen,
wobei \cref{sec:fazit:ziele} die erreichten Ziele zusammenfasst,
\cref{sec:fazit:kritik} Selbstkritik übt und schließlich \cref{sec:fazit:ausblick} einen Blick in die Zukunft wagt.

