\chapter{Einleitung (5\%) }

\section{Motivation}

Kontinuierliche Integration ist ein wachsender Bereich
in der Welt der Software-entwicklung und ihrer Qualit\"atssicherung.

% referenzen folwer/xp
Entstammend aus der Praxis des Extreme Programming
\cite{xp:explained} \cite{folwer:xp},
besch\"aftigen sich Werkzeuge f\"ur Kontinuierliche Integration damit,
Programme nach einer vollzogenen Integration zu Testen.

In regelm\"assigen Abst\"anden, wie z.B. nach einer erfolgreichen Integration oder nach Arbeitsschluss,
wird das Programm vobereitet und getestet.
Dies unterst\"utzt das rasche Auffinden von Fehlern
und informiert \"uber den Zustand der Software.
Nach Abschluss eines solchen Testes stehen dann das Resultat
und ein Statusbericht zur Verf\"ugung.

Im Laufe der Zeit hat sich der Inhalt der Statusberichte weiterentwickelt.
Waren es anf\"anglich nur sp\"arliche Daten \"uber Erfolg/Misserfolg,
so findet man heutzutage Mitschnitte verschiedener Datenquellen, Testergebnisse
\cite{jenkins:junitxml}, sowie detailierte Informationen \"uber die Umgebung des Testvorgangs.

Weiterhin entstand mit Zunehmenden Anforderungen an Portabilit\"at \"uber
verschiedendste Arten von Platformen (z.B. Datenbanksysteme oder Betriebssysteme)
die Notwendigkeit das System in mehr als nur einer Umgebung zu testen.

Auch die Verwaltung von Software hat sich ge\"andert.
W\"ahrend urspr\"unglich nur nur auf einem Entwicklungszweig gearbeitet und integriert wurde,
findet dies Dank moderner Quellcodeverwaltung jetzt in vielen Entwicklungszweigen statt.
\cite{dvcs:vorteile} \cite{dvcs:entwicklungsmodelle}

Aufgrund dieser Entwicklung stehen immer mehr Daten und Informationen zur Verf\"ugung,
und es erscheint \"uberaus sinnvoll, diese auch zu nutzen, um den Entwicklungsprozess zu unterst\"utzen.

Besonders Interessant sind dabei feststellbare Unterschiede zwischen den Testumgebungen,
sowie feststellbare Unterschiede zwischen verschiedenen Entwicklerzweigen.

%XXX: test vor integration?


\section{Problem}

Die Zukunftsvision, die Daten zu nutzen,
trifft in der Praxis auf einige harte Probleme.
Existierende Systeme sind schwer erweiterbar und haben oft schwer zugängliche Datenmodelle.

Es gibt keine Standardschnittstellen um CI-Systeme zu erweitern und mit ihren Daten zu arbeiten

Die Kombination dieser beiden Problemen erzeugt wiederum viele Probleme,
welche später im Kapitel \ref{chap:ist-analyse} behandelt werden.

Das Resultat ist, dass die existierenden Systeme im Bezug auf die Zukunftsvisionen,
nicht oder nur schwer anzupassen sind.

Der Mangel an Standardschnittstellen für den Datenzugriff und
das Fehlen von Standardschnittstellen für die Erweiterung der CI-Systeme
erschwert die Implementation von Analysewerkzeugen ungemein.



\begin{verbatim}
- datenbankferne
- probleme des client/server ?
\end{verbatim}

\section{Zielsetzung}

Ziel dieser Arbeit ist es  aufzuzeigen, dass die oben genannten Probleme lösbar sind.
Zu diesem Zweck soll die Grundlage für ein System geschaffen werden,
welches auch in der Lage ist dies umzusetzen.

Es ist somit ein grundlegendes verteiltes System f\"ur Kontinuierliche Integration zu schaffen,
welches auf verteilter Datenbanktechnologie basiert.
Dies dient als Ansatzpunkt, um das System zu erweitern und
Standardschittstellen f\"ur den Datenzugriff zur Verf\"ugung zu stellen.

Dabei sollten sowohl die grundlegenden Funktionen eines CI,
als auch die M\"oglichkeit der Erweiterung in Betracht gezogen werden,

Beispiele f\"ur solche Erweiterungen wären,
wie schon genannt,
vergleichende Analyse zwischen Entwicklungzweigen oder Eigenständige Analysewerkzeuge.

Wichtig ist hierbei die grundlegende Erweiterbarkit
und nicht die L\"osung eines speziellen Problemes.


\begin{verbatim}
- loesungswunsch
    - offene fragen
- dokumentorientert und verteilt
\end{verbatim}

\section{Abgrenzung}

\begin{verbatim}
- kernkomponenten
- prototyp
- ausgewaehlte erweiterung
  - praktisch analyse (sommer)
  - theoretish - regressions-vergleiche

- keine authentifikation
- keine user-interfaces

\end{verbatim}

\section{Aufbau der Arbeit}


