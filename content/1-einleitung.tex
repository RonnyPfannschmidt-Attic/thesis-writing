\chapter{Einleitung (5\%) }
\label{chap:intro}
\section{Motivation}

\ac{CI} ist ein wachsender Bereich
in der Welt der Software-Entwicklung und ihrer Qualitätssicherung.

% referenzen folwer/xp
Entstammend aus der Praxis des \ac{XP}
\cite{xp:explained, folwer:xp},
beschäftigen sich Werkzeuge für kontinuierliche Integration damit,
Programme nach einer vollzogenen Integration zu testen.

In regelmäßigen Abständen, wie zB. nach einer erfolgreichen Integration oder nach Arbeitsschluss,
wird das Programm vorbereitet und getestet.
Dies unterstützt das rasche Auffinden von Fehlern
und informiert über den Zustand der Software.
Nach Abschluss eines solchen Tests stehen dann das Resultat
und ein Zustandsbericht zur Verfügung.

Im Laufe der Zeit hat sich der Inhalt der Zustandsberichte weiterentwickelt.
Waren es anfänglich nur spärliche Informationen über Erfolg bzw. Misserfolg,
so findet man heutzutage Mitschnitte verschiedener Datenquellen, Testergebnisse
\cite{jenkins:junitxml}, sowie detaillierte Informationen über die Umgebung des Tests.

Weiterhin entstand mit zunehmenden Anforderungen an die Portabilität über
verschiedenste Arten von Plattformen (zB. Datenbanken oder Betriebssystemen)
die Notwendigkeit das System in mehr als nur einer Umgebung zu testen.

Die Verwaltung von Software hat sich geändert.
Während ursprünglich nur auf einem Zweig der Entwicklung gearbeitet und integriert wurde,
findet dies Dank moderner Quellcode-verwaltung jetzt in vielen Zweigen statt.
\cite{dvcs:vorteile, dvcs:entwicklungsmodelle}

Aufgrund dieser Entwicklung stehen immer mehr Daten und Informationen zur Verfügung,
und es erscheint überaus sinnvoll, diese auch zu nutzen, um den Entwicklungsprozess zu unterstützen.

%XXX woanders?
Besonders Interessant sind dabei feststellbare Unterschiede zwischen den Umgebungen,
sowie feststellbare Unterschiede zwischen verschiedenen Zweigen der Entwicklung.

%XXX: test vor integration?


%\begin{verbatim}
%start traditionelles ci
%ueberfuehrung anforderungen
%ueberfuehrung branches, build achsen
%groessere datenmengen
%analyse
%feststellung struktir
%\end{verbatim}



\section{Problem}

Die Zukunftsvision, die Daten zu nutzen,
trifft in der Praxis auf einige harte Probleme.
Existierende Systeme sind schwer erweiterbar und haben oft schwer zugängliche Datenmodelle.

Es gibt keine Standards für Schnittstellen um Systeme fur \ac{CI} zu erweitern und mit ihren Daten zu arbeiten

Die Kombination dieser beiden Problemen erzeugt wiederum viele Probleme,
welche später im Kapitel \ref{chap:ist-analyse} behandelt werden.

Das Resultat dieser Probleme ist, dass die existierenden Systeme im Bezug auf die Zukunftsvisionen,
nicht oder nur schwer anzupassen sind.

Der Mangel an Schnittstellen für den Datenzugriff und
das Fehlen von Schnittstellen für die Erweiterung der CI-Systeme
erschwert die Implementation von Werkzeugen für Analyse ungemein.


\section{Zielsetzung}
%XXX nochmal
Ziel dieser Arbeit ist es aufzuzeigen, dass die oben genannten Probleme lösbar sind.
Zu diesem Zweck soll der Kern für ein System geschaffen werden,
welches in der Lage ist dies umzusetzen.

Es ist somit ein verteiltes System für kontinuierliche Integration zu schaffen,
welches auf Datenbanktechnik basiert.
Dies dient als Ansatzpunkt, um das System zu erweitern und
Schnittstellen für den Datenzugriff zur Verfügung zu stellen.

Dabei sollten sowohl die grundlegenden Funktionen eines CI,
als auch die Möglichkeit der Erweiterung in Betracht gezogen werden.
Beispiele für solche Erweiterungen sind, wie schon genannt,
vergleichende Analyse zwischen Entwicklung-zweigen oder eigenständige Analysewerkzeuge.

Wichtig ist hierbei die grundlegende Erweiterbarkeit aufzuzeigen
und nicht die Lösung eines speziellen Problems.

\section{Abgrenzung}

Es soll lediglich der \emph{Kern} eines Systems als \emph{Prototyp} geschaffen werden,
damit werden Punkte wie die Benutzeroberfläche und Benutzer-verwaltung
außen vor gestellt.

Um die Arbeit überschaubar zu halten, wird dabei darauf verzichtet, Analysewerkzeuge außerhalb der verwendeten Datenbank zu betrachten.
Stattdessen wird der Kern des Systems und die Datenbank in den Fokus gestellt.


\section{Aufbau der Arbeit}


