\chapter{Thematische Einleitung}

\section{NoSQL}

NoSQL (auch Not only Sql) bezeichnet eine breite Klasse
von Datenbank Managementsystmen, welche sich im wesentlichen
vom traditionellen relationalen Model dadurch unterscheiden,
dass ihnen weder SQL als Abfragesprache, noch Joins zur Verfügung stehen.

Dabei h

\section{CouchDB}

Couchdb gehört zur den schemalosen Dokumentorientierten Datenbanken,
es legt Faten in Form von struktuerierten Dokumenten ab.

\subsection{Grundlgende Eigenschaften}

\begin{itemize}
\item integriertes map reduce
\item muti version currency controll
\item einfache Replikation/Kontinuierliche Replikation


\end{itemize}


\subsection{Views}

Views in Couchdb sind die Resultate des Integriertem MapReduce.
Sie werden geordnet nach dem Key abgelegt und koennen über Key Ranges abgefragt werden.


Views werden als Ersatz für die in Sql üblichen Indexe und Aggregationen verwendet.
Dabei stellt eine  reine map operation einen Index dar, und map+reduce eine aggragation

\newpage
Zum Beispiel
\lstsetjavascript
\begin{lstlisting}
{
    "_id": "user1",
    "type": "user",
    "name": "bob",
    "roles": ["admin", "developer"],
}

{
    "_id": "user2",
    "type": "user",
    "name": "alice",
    "roles": ["admin"],
}
\end{lstlisting}


Stellen 1 einfache Dokumene dar.

will man basierend auf dem Namen, suchen, so benötigt man eine Map View mit folgender Map Funktion
\begin{lstlisting}
function (doc) {
    if (doc.type == "user")
        emit(doc.name, null);
}
\end{lstlisting}

Im View befinden sich dann folgende Daten:

\begin{lstlisting}
{"total_rows":2,"offset":0,"rows":[
    {"id":"user2","key":"alice","value":null},
    {"id":"user1","key":"bob","value":null}
]}
\end{lstlisting}



\section{Kontinuierliche Integration/Verteilung}

\subsection{Überblick}

Kontinuierliche Integration/Verteilung bezeichent alle Prozesse,
welche dazu dienen automatisch Unittest, Integrations, und Akzeptanz tests Auszufuehren


\subsection{existierende Systeme}

\subsubsection{buildbot}

%XXX referenzen
BuildBot positioniert sich als eine Art Meta-Build-Server.
es biete keine normale Oberfläche, sondern wird mittels
Komposition von Metadaten und Komponenten konfiguriert.

Die Konfiguration des Servers stellt dabei ein python script,
welches die Operationen ausführt, welche zur Zusammenstellung des gewünschten Servers notwendig sind.

Es ist möglich Builds jobs mit minimalen parametern in form von strings zu uebergeben

\subsubsection{jenkins/hudson}

%XXX referenzen
Jenkins und Hudson stellen ein Benutzerfreundliches,
jedoch limitiertes System zum einfachen Anlegen von Build-Jobs.

Parametrisierung ist nicht möglich.
