\chapter{Thematische Einleitung}

\section{NoSQL}


NoSQL (auch Not only Sql) bezeichnet eine breite Klasse
von Datenbank Managementsystmen, welche sich im wesentlichen
vom traditionellen relationalen Model dadurch unterscheiden,
dass ihnen weder SQL als Abfragesprache, noch Joins zur Verfügung stehen.
~ \cite{wikipedia:nosql}


Dabei h

\section{CouchDB}

Couchdb gehört zur den schemalosen dokumentorientierten Datenbanken,
es legt Daten in Form von strukturierten Dokumenten ab.

\subsection{Grundlgende Eigenschaften}


\begin{itemize}
\item Geschrieben in: Erlang
\item Hauptargumente: Lonsistenz, einfache Verwendung
\item Lizenz: Apache
\item Protocol: HTTP/REST
\item Bi-directionale (!) replication,
    continuierlich or ad-hoc,
    with Konflikt Erkennung,
    was bedeutet, master-master replication. (!)

\item MVCC - schreib operationen blockieren leseoperationen nicht
\item alte Dokumentrevisionen verfuegbar
\item Crash-only (reliable) design
\item benötigt regelmässige Kompaktion/Optimierung
\item Views: eingebautes map/reduce
\item Formatierende Vews: lists \& shows
\item Server-seitige Document validation möglich
\item Authentication possible
\item Real-time updates via \_changes (!)
\item Attachment handling
\item thus, CouchApps (standalone js apps)
\item jQuery library included 
\end{itemize}
~ \cite{web:db-compare}

Best used: For accumulating, occasionally changing data, on which pre-defined queries are to be run. Places where versioning is important.

For example: CRM, CMS systems. Master-master replication is an especially interesting feature, allowing easy multi-site deployments. 


\begin{itemize}
\item integriertes map reduce
\item muti version currency controll
\item einfache Replikation/Kontinuierliche Replikation
\end{itemize}


\subsection{Grundlagen Views}
\subsection{Indizierung mit Views}
\subsection{Aggregation mit Views}
\subsection{Join mit Views}



Views in Couchdb sind die Werkzeuge,
welche das Pendant zu Indexen und Aggregationen
aus traditionellen Datenbanksystemen stellen.

%XXX satzbau
Im Gegensatz zu Indexen in traditionellen Datenbanken, werden sie auf alle Dokumente,
nicht nur die Dokumente eines Speziellen Types angewendet


Resultate des Integriertem MapReduce.
Sie werden geordnet nach dem Key abgelegt und koennen über Key Ranges abgefragt werden.


Views werden als Ersatz für die in Sql üblichen Indexe und Aggregationen verwendet.
Dabei stellt eine  reine map operation einen Index dar, und map+reduce eine aggragation

\newpage
Zum Beispiel
\lstsetjavascript
\begin{minted}{javascript}
{
    "_id": "user1",
    "type": "user",
    "name": "bob",
    "roles": ["admin", "developer"],
}

{
    "_id": "user2",
    "type": "user",
    "name": "alice",
    "roles": ["admin"],
}
\end{minted}


Stellen 1 einfache Dokumene dar.

will man basierend auf dem Namen, suchen, so benötigt man eine Map View mit folgender Map Funktion
\begin{lstlisting}
function (doc) {
    if (doc.type == "user")
        emit(doc.name, null);
}
\end{lstlisting}

Im View befinden sich dann folgende Daten:

\begin{lstlisting}
{"total_rows":2,"offset":0,"rows":[
    {"id":"user2","key":"alice","value":null},
    {"id":"user1","key":"bob","value":null}
]}
\end{lstlisting}



\section{Kontinuierliche Integration/Verteilung}

\subsection{Überblick}

Kontinuierliche Integration/Verteilung bezeichent Programme/Dieste,
welche dazu Dienen automatisch 

\subsection{existierende Systeme}

\subsubsection{buildbot}

%XXX referenzen
BuildBot positioniert sich als eine Art Meta-Build-Server.
es biete keine normale Oberfläche, sondern wird mittels
Komposition von Metadaten und Komponenten konfiguriert.

Die Konfiguration des Servers stellt dabei ein Python script,
welches die Operationen ausführt, welche zur Zusammenstellung des gewünschten Servers notwendig sind.

Es ist möglich Builds Jobs mit minimale Parameter in Form von Strings zu Übergeben.

\subsubsection{jenkins/hudson}

%XXX referenzen
Jenkins und Hudson stellen ein Benutzerfreundliches,
jedoch limitiertes System zum einfachen Anlegen von Build-Jobs.

Parametrisierung ist nicht möglich.

\subsection{Vor- und Nachteile der existierenden Systeme}
