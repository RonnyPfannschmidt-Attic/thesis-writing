\chapter{Einleitung (5\%) }

\section{Motivation}

In den letzten Jahren gewannen die NoSQL Datenbanken immer mehr an Bedeutung.
Die Abkehr von SQL, ACID und Relationen, sowie die Verschiebung von Konsistenz zu eventueller Konsistenz, um Partitions-Toleranz zu maximieren,
benötigt eine Grundlegend andere Herangehensweise im Aufbau von Applikationen und Diensten.

Diese Unterschiede fanden bisher kaum Beachtung.

Ein weiteres, davon unabhängiges Problem,
ist der aktuelle Zustand der Werkzeuge für Kontinuierliche Integration.

Viele existierende Werkzeuge sind entweder einfach gestrickt,
oder nur schwer einzurichten.

Weiterhin sind die Systeme in der Regel nicht oder nur Umständlich zu erweitern.

Es bestehtl also ein Bedarf an Werkzeugen welche sowohl einfach einzurichten,
als auch einfach zu erweitern sind.

Die Möglichkeiten von NoSQL Datenbanken erscheinen geeignet, diese beiden Ziele zu unterstützen.
Schemalosigkeit, eingebaute Funktionen für verteiltes Arbeiten,
sowie die Grundlegend anderen Herangehensweisen erscheinen
als vielversprechende Grundbausteine, deren Nutzen es noch Darzustellen gilt.




\section{Problemstellung}

Um ein besseres Verständnis für die Unterschiede bei Modellierung zu gewinnen,
sollen diverse Teile eines Gesamtsystems für die Kontinuierliche Integration einzeln Analytisch und/oder praktisch gegenüber gestellt werden.
Anschliessend sollen diese Anhand der in der Zielsetzung definierenden Kriterien verglichen werden.

Weiterhin soll am Beispiel aufgezeigt werden, wie einfach bestimmte Werkzeuge,
die in aktuellen Systemen zur kontinuierlichen Integration nicht oder nur schwer umgesetzt werden können,
mittels der Werkzeuge, welche in der für die Praxis-Beispiele gewählten NoSQL Datenbank vorhanden sind,
umgesetzt werden können.

Abschliessend soll die Implementations-Komplexität von Standardaufgaben verglichen werden.


\section{Zielsetzung}

Die vergleichende Bewertung von grundverschiedenen Herangehensweisen stellt vor ein Grundlegendes Problem.

F\"ur den Vergleich bzw. die Gegen\"uberstellung von Programm-Designs gibt es kein Gundlegendes Formales Mittel,
um Objektv zu Vergleichen.

Der Inneliegende Semantische Kontext macht eine interpretation der Unterschiede
in den formalen Beschreibungen der einzelnen Designs notwendig.
Weshalb der Vergleich eine Grundlegende subjektive Komponente enth\"alt.

Diese subjektive Komponente ist im Rahmen dieser Arbeit jedoch W\"unschenswert,
da eine Perspektve n\"aher an das Thema heranf\"uhrt.
Letztendlich ist Programmdesign und Entwicklung trotz aller Objektivit\"at
doch abh\"angig von den Erfahrungen und Einsichten des menschlichen Entwicklers.
Somit bietet die Eigene Perspektive dem Leser einen Ansatzpunkt f\"ur Verst\"andniss und Kritik.





\section{Abgrenzung}


Die variantenreiche Landschaft der NoSQL Datenbanken ist reich an unterschiedlichen Lösungen,
welche selbst Untereinander gravierende Unterschiede in Speicherung, Abfrage und Aggregation von Daten aufweisen.

Daher ist eine Gesamt-Bearbeitung weit \"uber dem Rahmen einer Diplomarbeit.
Stattdessen sollen die Differenzen am Beispiel einer noch Auszuw\"ahlenden Datenbank
Theoretisch und Beispielhaft dargestellt werden, um eine Bewertung zu erm\"oglichen.


\section{Aufbau der Arbeit}


Zu Beginn wird in Kapitel 2 die Zentrale Fragestellung der Arbeit geklärt,
Ziel der Arbeit ist es diese Frage umfassend zu beantworten.

Um dies zu ermöglichen, wird in Kapitel 3 ein grundlagenwissen aufbereitet,
Zuerst wird dabei in Abschnit 3.1 auf Datenbanken im allgemeinen eingegangen

Fortsetzend wird dann in Kapitel 3.2 auf SQL
und die grundlegenden Eigenschaften des relationalen Systems noch einmal dargestellt.

Anschliessend wird in Abschnitt 3.3 auf Nosql eingegangen
und ebenfals die grundlegenden Eigenschafen erläutert.

Der Abschnitt 3.4 wir dazu verwendet um einen groben Überblick über Kontinuierliche Integration zu schaffen.

Letzendlich werden dann ab Abschnitt 3.5 Grundlagen und Technicken zur später
im Praktischen Teil ausgewählten Datenbank Couchdb erläutert.

% views? replicaton ?



Kapitel 4 widmet sich Theoretischen Betrachtungen.
in Abschnitt 4.1 werden Vergleiche von Grundlegenden Herangehensweisen
getroffen.

Anschliessend wird in Abschnitt 4.2 der Gundlegende Aufbau eines einfachen CI Systemes
entwickelt, welches für die Praktischen Gegenüberstellugen verwendet wird.

In Kapitel 5 wird schiesslich Detailiert herausgearbeitet,
wie das Experimentiersystem aufgebaut ist, und warum.

Die Abschnitte des 6. Kapitels werden je einen Teilaspekt des Beispielsystems,
gegenüberstellend analysieren und anschliessend prototypisch implementieren,
sowie das erarbeitete wissen kurz zusammenfassen.

Darauffolgend werden in Kapitel 7 Prototypen für unübliche Funktionalitäten geschaffen,
welche jedoch f\"ur den Endnutzer einen Grossen Mehrgewinn bedeuten.

In Kapitel 8 werden dann Prototypen fuer Standardaufgaben geschaffen,
und deren Komplexitaet/aufwand mit den existerenden Werkzeugen.

Anschliessend wird in Kapitel 9 ein Gesammtfazit geschaffen und Selbstkritik ge\"ubt,
sowie Ideen f\"ur Weiterf\"uhrende Arbeiten vorgestellt.
