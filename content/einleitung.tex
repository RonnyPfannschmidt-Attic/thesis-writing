\chapter{Einleitung (5\%) }

\section{Motivation}

In den letzten Jahren gewannen die NoSQL Datenbanken immer mehr an Bedeutung.
Die Abkehr von SQL, ACID und Relationen, sowie die Verschiebung von Konsistenz zu eventueller Konsistenz, um Partitions-Toleranz zu maximieren,
benötigt eine Grundlegend andere Herangehensweise im Aufbau von Applikationen und Diensten.

Diese Differenzen sind bisher wenig Dokumentiert.

\section{Problemstellung}

Um ein besseres Verständnis für die Unterschiede bei Modellierung zu gewinnen,
sollen diverse Teile eines Gesamtsystems für die Kontinuierliche Integration einzeln Analytisch und/oder praktisch gegenüber gestellt werden.
Anschliessend sollen diese Anhand der in der Zielsetzung definierenden Kriterien Verglichen werden.

Weiterhin soll am Beispiel Aufgezeigt werden, wie einfach bestimmte Werkzeuge,
die in aktuellen Systemen zur Kontinuierlichen Integration nicht oder nur schwer umgesetzt werden können,
mittels der Werkzeuge, welche in der für die Praxis-Beispiele Gewählten NoSQL Datenbank vorhanden sind,
umgesetzt werden können.



\section{Zielsetzung}




\begin{verbatim}
- bewerungskriterien fuer vergleich der exemplarischen informationen


- vergleichende modellierung
  - projekte modelieren
  - integrationen modellieren
  - states von integrationen
  - jobs + ihre states
  - job fencing
  - verteilungsmodelle

- modellierung neue tools use-cases
  - send workdir diff/tarball
  - auswertung zeitserien

- kombination der exemplarischen implementationen


\end{verbatim}

\section{Abgrenzung}


Die variantenreiche Landschaft der NoSQL Datenbanken ist reich an unterschiedlichen Lösungen,
welche selbst Untereinander gravierende Unterschiede in Speicherung, Abfrage und Aggregation von Daten aufweisen.

Daher ist eine Gesamt-Bearbeitung weit Über dem Rahmen einer Diplomarbeit.
Stattdessen sollen die Differenzen am Beispiel einer noch Auszuwählenden Datenbank
Theoretisch und Beispielhaft dargestellt werden, um eine Bewertung zu ermöglichen.


\section{Aufbau der Arbeit}




\begin{verbatim}`
- grundlagen
  - datenbanken
  - sql
  - nosql
  - die ausagewaehlte datenbank couchdb
    - views
    - replikation

- theorie
  - vergleiche grundlegender herangehensweisen
  - aufbau eines einfachen ci systems

- praxisablauf je beispiel
    - semantik relational (evtl beispiel snippets)
    - semantik nosql
    - prototyp
    - fazit

- gesammtfazit
    - zusammenfassung einzelfazite
    - selbstkritik
- ideen fuer weiterfuehrungen/fortsetzungen
\end{verbatim}
