\chapter{Evaluation \& Optimierung ( 25 \%)}

\section{Testfälle}
\subsection{Grundlegende Werkzeuge}

\begin{verbatim}
- pytest erwaehnen?
- upload tool erwaehnen
\end{verbatim}

\subsection{Grundlegende Primitiven}

\begin{verbatim}
- watch-for -> funktioniert
- run_callbacks
- gather next #XXX optimierung
\end{verbatim}

\subsection{Auftragsannahme}
\begin{verbatim}
- matrizen
- validation
\end{verbatim}

\subsection{Auftragsverwaltung}
- verteilung
- konflikte?

\subsection{Auftragsabarbeitung}

\subsection{Prozessschritte}
\subsection{Quellcode Management Schritte}

\begin{verbatim}
- checkout
- upddate
\end{verbatim}

\subsection{Exemplarische Erweiterung - Datenaggregation}

\subsection{Zeitanforderungen}

\section{Festgestellte Probleme}
% auffaelliges zeitverhalten
\subsection{Anlaufphase}

\begin{verbatim}
- beim start mit grosser auftrags/datenmenge
- zeit steigt linear
- 

\end{verbatim}

\subsection{Konfliktsituation Arbeitspackete anfordern}

\begin{verbatim}
- fiel beim
\end{verbatim}



\subsection{Locksituation Abarbeitung von Managementdaten}

\section{Optimierung}
% beschreiben dass man eigentlich ab kap 3 iterieren muesste
% als abgrenzung wird das ganze prototypisch geloest

\subsection{Anlaufphase}
\subsubsection{Analyse}


\subsubsection{L\"osungsansatz}

\subsubsection{Implementation}


% umsortiern
\subsection{Konfliktsituation Arbeitsackete}
\subsubsection{Analyse}
\subsubsection{L\"osungsans\"atze}
\subsubsection{L\"osungsauswahl}
%- implementationstechnik beschreiben
% konzeption gleich

\subsubsection{Implementation}

\subsection{Theorie Locksituation Managementdaten}
\subsubsection{Analyse}
% ausblick weil sonst rahmen gesprengt
\subsubsection{L\"osungsans\"atze}
\subsubsection{L\"osungsauswahl}
