\chapter{Evaluation}
\section{Testumgebung}

\subsection{Manuelle Tests}
Um das Gesamte System zu testen, wurde der Physische Überblick (siehe \cref{fig:grob-layout-komponenten}) als Plan hergenommen.
Und auf 3 Rechnern (für jeden Knoten der Datenbank einen) umgesetzt.
Dabei wurde auf jedem Rechner für jede CPU ein Arbeiter gestartet.
zusätzlich wurde Willkürlich auf einem der Rechner der Manager gestartet.

Um Daten an das Testsystem zu bringen wurde dabei das \verb|put.py| Tool mit beispielhaften Daten verwendet.
Zur vorbereiten der Datenbank wurde jeweils couchdb-compose \cite{couchdb:compose} verwendet.
Anschließend wurde das System beobachtet.

\subsection{Automatische Tests}
Die Automatischen Tests wurden unter Zuhilfenahme von pytest \cite{pytest:website} und seiner Erweiterung  pytest-couchdb \cite{pytest:couchdbkit} auf dem Entwicklungsrechner  ausgeführt.
Dazu wurde lediglich eine einzelne Datenbank verwendet.
Die automatischen Tests bereiten die Datenbank dabei vor jeder Ausführung der Tests selbstständig vor.

\section{Testfälle}

Diese Sektion beschreibt die einzelnen komplexen Tests des Systems.

\subsection{Funktionale Tests}
Die Funktionalen Tests stellen die Funktionsweise einzelner Komponenten sicher,
Komponenten die im Prototypen nur eine einzige Aufgabe haben,
erhielten daher auch nur einzelne Tests, während Andere Komponenten mehrere Tests erhielten.

Die Tests verteilen sich Dabei wie folgt:

\begin{description}
    \item[Eingang] \hfill
    \begin{itemize}
        \item Einfache Validation
        \item Arbeitspakete ohne eigene Achsen generieren
        \item Arbeitspakete mit eigenen Achsen generieren
    \end{itemize}
    \item[Management] \hfill
    \begin{itemize}
        \item Arbeitsschritte vom Template generieren
        \item Arbeitspaket Zuteilung Konfliktfrei
        \item Arbeitspaket Zuteilung Konflikt
    \end{itemize}
    \item[Arbeiter] \hfill
    \begin{itemize}
        \item Arbeitspaket anfordern Konfliktfrei
        \item Arbeitspaket anfordern Konflikt
        \item Zugewiesenes Arbeitspaket Ausführen
    \end{itemize}
    \item[Arbeitsschritte] \hfill
    \begin{description}
        \item[Prozess] \hfill
        \begin{itemize}
            \item Einfacher Prozess
            \item Einfacher Prozess Fehlschlag
            \item Prozess mit vielen ausgaben (cat)
            \item Python Prozess
        \end{itemize}
        \item[SCM] \hfill
        \begin{itemize}
            \item Clone lokal
            \item Clone remote
            \item Clone Fehlschlag
        \end{itemize}
    \end{description}
\end{description}

\subsection{Systemtests}

Von den 3 Systemtests wurden nur 2 Automatisiert.
die Tests mit dem Durchlauf der Funktionalen  und des Gesamtsystems
zeigten viele kleine Fehler auf die hier nicht Näher betrachtet werden.

Das Testen der Beispielerweiterung stellte sich als kompliziert heraus, und wurde auch als Manueller Test umgesetzt.

Der Durchlauf der Komponenten führt einfach nur die Komponenten der logischen Reihenfolge nach aus.
\begin{enumerate}
    \item Auftrag Überprüfung
    \item Auftrag Vorbereitung
    \item Auftrag Arbeitspakete Generieren
    \item Arbeitspaket Schritte Generieren
    \item Arbeitspaket Inanspruchnahme
    \item Arbeitspaket Zuweisung
    \item Arbeitspaket Erwarten
    \item Arbeitspaket Ausführen
\end{enumerate}

Der Test des Gesamtsystems verwendet die beschriebenen Systemkomponenten, um den Manager und 2 Arbeiter zu starten und gibt Aufträge ins System

\subsection{Manuelle Tests}

\subsection{Exemplarische Erweiterung - Datenaggregation}



\section{Festgestellte Probleme bei den manuellen Tests}

Bei den manuellen Tests, welche zum händischen Analysieren
einiger schwer testbarer Problem-punkte dienen,
sind einige schwerwiegende Probleme aufgetreten,
dieser werden hier nur kurz umrandet.

\begin{description}
    \dhitem[Langwierige Anlaufphase]
        Beim wiederholten Testen mit der gleichen Datenbank fiel auf, dass die Menge der Daten in der Datenbank Einfluss auf das Starten von Komponenten hat. Je mehr Daten sich in der Datenbank befinden, desto Länger benötigen Komponenten, um mit ihrer Arbeit zu beginnen.
    \dhitem[Konfliktsituation Inanspruchname Arbeitspackete]
        Beim Testen mit vielen Arbeitern fiel auf, dass immer alle gerade freien Arbeiter das selben nächsten Arbeitspaket in Anspruch nehmen wollen. Dies führte oft dazu, das manche Arbeiter erst nach Dutzenden Versuchen das nächsten Arbeitspaket bearbeiten können.
    \dhitem[Blockade Abarbeitung von Managementdaten]
        Beim Absetzen großer Aufträge fiel auf, dass der Manager in einigen Situationen für mehrere Minuten keine Inanspruchnahme eines Arbeitspaketes bearbeitete. Somit wurden Arbeitern keine Arbeitsschritte gegeben, obwohl diese eigentlich verfügbar waren.
    \dhitem[Blockade Datenausgabe Python]
        Bei der Beobachtung des Systems fiel auf, dass Python Schritte/Prozesse
        die Daten von STDOUT/STDERR nicht kontinuierlich und Zeilenweise,
        sondern in größeren Blöcke zur Ausgabe schickten.
\end{description}

\chapter{Optimierung}

Nachdem die Probleme festgestellt wurden
müsste eigentlich ab \Cref{chap:ist-analyse} neu iteriert werden,
um die Anforderungen entsprechen anzupassen.
Um überhöhten Aufwand zu vermeiden,
sollen die Problemlösungen stattdessen hier vorgestellt werden
und eine Auswahl auch Implementiert werden.

\section{Langwierige Anlaufphase}

\subsection{Analyse}

Das Problem der Anlaufphase lässt sich mit dem Verwenden der Mitteilungen über Änderungen erklären. Der Datenstrom beginnt dabei immer mit der ersten Änderung,
somit steigt die Menge an Änderungen, welche nicht betrachtet werden, aber dennoch Teil der Mitteilungen sind, mit der Datenmenge in der Datenbank an.
Dies führt zwangsläufig dazu, dass die Primitiven \verb|watch_for| und \verb|listen_changes| mehr Zeit benötigen, um zur ersten verwertbaren Mitteilung vorzudringen.


\subsection{Lösungsansatz Zustand Speichern}

Da es Möglich ist die Mitteilungen über an einem bestimmten Änderungsstand der 
Datenbank beginnen zu lassen (update sequence),
kann durch Speichern und Wiederverwenden der aktuellen Sequenz verhindert werden,
dass das System auch die bereits fertiggestellten Änderungen betrachten muss.
Dazu werden neue Datenobjekte benötigt, welche dies für die einzelnen Arbeiter festhalten. Eine Überarbeitung ab Grobkonzept wäre Notwendig


\subsection{Lösungsansatz CouchDB View}

Ein View in CouchDB ist ein der Lage Daten zu einem bestimmten Stand der Datenbank abzubilden und dieses Abbild zu Speichern.
Damit ist es Möglich den aktuellen Zustand sowie den Stand der Änderungen zu erfragen und somit den Zustand des Views, sowie die darauf-folgenden Änderungen zu bearbeiten 
Dies benötigt nur eine Überarbeitung der Techn. Konzeption und der Implementation.

\subsection{Implementation}
Da der View \verb|stm| bereits die Anforderungen des View-Ansatzes erfüllt,
wurde er zur Implementation verwendet.

Die primitiven \verb|watch_for| und \verb|run_callbacks| wurden modifiziert zuerst
Die Daten aus dem View \verb|stm| zu bearbeiten und anschließend 
auf nachfolgenden Änderungen zuzugreifen.

\subsection{Resultat}
Das Problem ist damit eingestellt, das System hat keinerlei Wartezeiten mehr bei der Anlaufphase.


\section{Konfliktsituation Inanspruchnahme Arbeitspakete}
\subsection{Analyse}
Die Konflikte bei der Inanspruchnahme von Arbeitspaketen ergeben sich aus der linearen Abarbeitung der Mitteilungen über Änderungen.
Jeder gerade freie Arbeiter hat die selbe Ansicht auf das nächste zur Verfügung stehende Arbeitspaket, deshalb fordern alle gleichzeitig dieses auch an.
Jedoch darf nur ein Arbeiter das Arbeitspaket auch bearbeiten,
alle anderen müssen es erneut versuchen.

\subsection{Lösungsansatz Token}
Der Lösungsansatz Token bedingt, dass die Auftragsvergabe auf das im Grobkonzept besprochene Melde-verfahren umgestellt wird.
Es gib dann nicht mehr die Möglichkeit, dass Arbeiter überhaupt Anspruch auf Arbeitspakete erheben, somit existiert der Konflikt auch nicht mehr.
Jedoch müssen dabei auch die Nachteile dieses Verfahrens in Kauf genommen werden.

\subsection{Lösungsansatz Zufall}

Der Lösungsansatz ``Zufall'' soll eine Zufallsfunktion dazu verwenden,
eines von mehreren verfügbaren Arbeitspaketen auszuwählen.
Durch die Steigerung der Menge kann dabei die Wahrscheinlichkeit des Konfliktes bei einzelnen Arbeitspaketen verringert werden. Dabei wird die Gleichverteilung eingesetzt, welche jedem Element in der Menge die gleiche Wahrscheinlichkeit gibt.

Im Rahmen dieser Arbeit stellte sich dieser Ansatz als Implementationstechnik heraus.


\subsection{Implementation}

Aufgrund der vorhergehenden Optimierung für die Anlaufphase,
bietet die primitive \verb|watch_for| einen geeigneten Ansatzpunkt.
Anstelle des nächsten Objektes im View, werden die nächsten $N$ Objekte erfragt,
und anschließend Zufällig eines davon ausgewählt.
Dabei wurde in den Tests $N = 10$ gewählt.

\subsection{Resultat}
Im normalen Ablauf sind Konflikte nicht länger bemerkbar.
Sie sie sind nur noch feststellbar, wenn die Menge der verfügbaren Arbeitspakete
sehr gering ist oder bereits mehrere Arbeiter dazu übergegangen sind,
auf eine Mitteilung über einen neuen Auftrag zu warten.

Dies bedeutet, das Konflikte nur noch dann auftreten,
wenn Arbeiter mangels weiterer verfügbarer Arbeitspakete nicht mehr in der Lage sind weiter zu Arbeiten. Damit kann das Problem als gelöst betrachtet werden.

\section{Theorie Blockade Managementdaten}
\subsection{Analyse}
Das Blockieren bei der Bearbeitung von Managementdaten lässt sich
aus der Kombination von 2 Umständen erklären.
Zum einen gibt die Primitive ``listen\_changes'' Mitteilungen
in der Reihenfolge der Objekt-Erstellungen/-Änderungen zurück.
Zum anderen die Zusammenlegung von Zuteilung, Vorbereitung
und Erstellung von Arbeitspaketen.
Wird eine große Menge an Arbeitspaketen generiert,
so ist das Management anschließend lange Zeit damit beschäftigt,
diese für die Abarbeitung vorzubereiten.

In dieser Zeit kann Aufgrund der Abarbeitungsreihenfolge
keine anderen Aufgaben wahrnehmen.
Somit werden andere Vorgänge blockiert.

\subsection{Lösungsansatz Prozess-Spaltung}

Der Ansatz der Spaltung teilt die Bearbeitung verschiedener Callbacks im Manager
in verschiedene Prozesse, jeder Prozess bearbeitet dabei einem eigenen Strom an Mitteilungen, dies macht die verschiedenen Operationen unabhängig und neben-läufig.
Somit kann das Bearbeiten einer Art von Änderung nicht länger das Bearbeiten einer anderen beeinflussen 

Die Aufteilung in mehr Prozesse benötigt eine Umstellung Der Technischen Konzeption und der Implementation

\subsection{Lösungsansatz Priorisieren}

Bei der Priorisierung von Notifikationen ist das Ziel,
die Notifikationen welche Einfluss auf andere Komponenten haben zuerst zu betrachten.
Inanspruchnahme sollte also immer vor anderen Notifikationen bearbeitet werden.
Dies würde Bedeuten, das Vor/Nachbereitung nicht länger die Zuteilung von Arbeitspaketen behindern.

Um dies umzusetzen wäre eine Neukonzeption im Grobkonzept notwendig.


\subsection{Ausgrenzung}

Aufgrund der relativen Komplexität der Umsetzung,
wurde darauf verzichtet die vorgestellten Lösungsansätze der Locksituation Management
als Teil dieser Arbeit zu testen.

\section{Blockade Datenausgabe Python}
\subsection{Analyse und Implementation}

Bei Recherchen im Internet stellte sich heraus,
dass Python in der Standardeinstellung die Datenausgabe puffert und somit Zeilen nicht direkt,
sondern in großen Blöcken ausgibt.
%XXX citation
Die Quellen legten nahe, den Interpreter einfach nur so zu Konfigurieren,
dass er nach jeder Ausgabe auch den Puffer in den Datenstrom ausgibt.

\subsection{Resultat}

Die Ausgaben werden nun Zeilenweise in das System gegeben,
es entstehen keine Blockaden mehr.