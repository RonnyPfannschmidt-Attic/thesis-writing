\chapter{Evaluation \& Optimierung ( 25 \%)}

\section{TestFaelle}
\subsection{Grundlegende Werkzeuge}
\subsection{Grundlegende Primitiven}
\subsection{Auftragsannahme}
\subsection{Auftragsverwaltung}
\subsection{Auftragsabarbeitung}
\subsection{Prozessschritte}
\subsection{Quellcode Management Schritte}
\subsection{Exemplarische Erweiterung - Datenaggregation}


\section{Festgestellte Probleme}
% auffaelliges zeitverhalten

\subsection{Anlaufphase}
\subsection{Konfliktsituation Arbeitspackete anfordern}
\subsection{Locksituation Abarbeitung von Managementdaten}

\section{Optimierung}
% beschreiben dass man eigentlich ab kap 3 iterieren muesste
% als abgrenzung wird das ganze prototypisch geloest



\subsection{Anlaufphase}
\subsubsection{Analyse}
\subsubsection{L\"osungsansatz}
\subsubsection{Implementation}


% umsortiern
\subsection{Konfliktsituation Arbeitsackete}
\subsubsection{Analyse}
\subsubsection{L\"osungsans\"atze}
\subsubsection{L\"osungsauswahl}
%- implementationstechnik beschreiben
% konzeption gleich

\subsubsection{Implementation}

\subsection{Theorie Locksituation Managementdaten}
\subsubsection{Analyse}
% ausblick weil sonst rahmen gesprengt
\subsubsection{L\"osungsans\"atze}
\subsubsection{L\"osungsauswahl}
