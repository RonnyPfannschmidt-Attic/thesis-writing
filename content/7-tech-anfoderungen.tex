\chapter{Tecnologische Konzeption ( 5 \% )}


% begruendet technologie, ist trasnformation notwendig

% evtl kap 7 referenzieren


\section{Datenbank}
\subsection{Technologische Betrachtung}

\begin{verbatim}
- reliational vs nosql (documentorientiert)
    - eignung aller mit div eigenschaften
- verteilte datenbank
\end{verbatim}


\subsection{Producktanforderungen}

\begin{verbatim}
- eingrenzung auf master master replikation
- selektive/gefilterte replikation wuenschenswert
- change notification
- local atomic

-> modellierung referenierung
% schemata
% mapping auf datenbank aenderungen

-> couchdb ausarbeiten

\end{verbatim}


\section{Programmiersprachen}
\subsection{Technologische Betrachtung}

\begin{verbatim}
- scriptsprachen f produktivitaet
- ggv db interne sprache
\end{verbatim}
% klassen/moduldiagramme aus kap 5 referenzieren

\subsection{Produktanforderungen}
%-> in die implementation

\begin{verbatim}
- ist python/warum python
\end{verbatim}


\section{Hilfswerkzeuge}
\subsection{Technologische Betrachtung}
\begin{verbatim}
- tdd
- 
\end{verbatim}
\subsection{Produktanforderungen}
%-> in die implementation
\begin{verbatim}
- pytest ref
\end{verbatim}
