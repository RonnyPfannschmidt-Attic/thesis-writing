\chapter{Technologische Konzeption ( 5\% )}


Im vorhergehenden Kapitel wurde das gundlegende System konzipiert.
Fortsetzend daran sollen nun geeignete Technologien und Werkzeuge gefunden werden,
um einen Prototyp umzusetzen.
Ausserdem werden notwendige Anpassungen der Konzepte an technische Gegebenheiten vorgenommen werden,
um die fortf\"uhrende Implementation zu unterst\"utzen.


% begruendet technologie, ist trasnformation notwendig

% evtl kap 7 referenzieren


\section{Datenbank}

Die Auswahl der Datenbanktechnologie und eines Datenbankproduktes sind die wichtigsten technologischen Entscheidungen dieser Arbeit.
Die im Kapitel~\ref{chap:design} vorgestellten L\"osungen.
Die Datenbank stellt die Basis des Systems und ihre Eigenschaften werden die Implementation massgeblich beeinflussen.


\subsection{Technologische Betrachtung}

%XXX: Bild overview verweisen
In einer initialen Sondierung zeigte sich,
dass 3 Technologieklassen die grundlegenden Anforderungen an die Datenmodellierung erf\"ullen.
Dies sind die Dokumentorientierten, die Graphendatenbanken und moderne relationale Datenbanken.

\subsection{Produktanforderungen}

Wie die technologische Betrachtung zeigte,
sind mehrere Arten von Datenbank f\"ur die Modellierung geeignet.

Um ein geeignetes Produkt zu bestimmen,
m\"ussen Anforderungen herausgearbeitet werden,
welche sp\"ater bei der Implementation hilfreich sein k\"onnten.

Wichtigstes Kriterium ist dabei, dass die Datenbank tats\"achlich \textbf{verteilt} ist.
Um sicherzustellen, das alle Systemanforderungen erf\"ullt werden k\"onnen,
muss die Datenbank zus\"atzlich \textbf{Master-Master Replikation} beherschen.
Selbstverst\"andlich muss die Replikation mit einer \textbf{partitionierung} des Netzwerkes zurechtkommen.

Ein weiteres wichtiges Werkzeug, welches die Datenbank mitbringen sollte, sind \textbf{\"Anderungs-Mitteilungen}.
Unterschiedliche Komponenten des Systemes beginnen mit ihnren Aufgaben, wenn \"Anderungen an der Datenbank stattfinden.
Daher sind detailierte Echtzeit-Informationen \"uber \"Anderungen der Datenbank eine entscheidende Komponente,
deren Einfluss auf Latenzen und Antwortzeiten der Komponenten nicht zu untersch\"atzen ist.


%XXX evtl db management
\begin{verbatim}

- selektive/gefilterte replikation wuenschenswert
- local atomic <> mvcc ?

-> modellierung referenierung
% schemata
% mapping auf datenbank aenderungen

-> couchdb ausarbeiten

\end{verbatim}


\section{Programmiersprachen}

Die Programmiersprachen stellen die Basis f\"ur produktives Entwicken,
sie erm\"oglichen erst die Umsetzung eines Projektes.

\subsection{Technologische Betrachtung}

Da ein Prototyp geschaffen werden soll,
ist eine Pogrammiersprache mit hoher Produktivit\"at gefragt,
selbst wenn dies auf Konsten von Performance geschiet.

Daher bietet sich die Klasse der Scriptsprachen an,
ihre dynamische Natur erm\"oglicht es,
in kleineren Projekten schnell zum Ziel zu kommen.
%XXX cite?
Weiterhin muss eventuell die Datenbank-interne Sprache beachtet werden.

% klassen/moduldiagramme aus kap 5 referenzieren

\subsection{Produktanforderungen}

Im Detail sind nur einige wenige Punkte zu beachten.
Die Wahl der Sprache an sich steht frei,
solange bestimmte Bibliotheken verf\"ugbar sind.

Wichtig sind dabei vor allem folgende Punkte.
\begin{itemize}
    \item Datenbankzugriff 
    \item Prozesskontrolle f. Prozess basierte Arbeitsschritte
    \item Zugriff Versionskontrolle
    \item Testing
\end{itemize}

Ausserdem sollten der Entwickler bereits mit der Sprache vertraut sein.

F\"ur die Datenbank-interne Sprache muss sich nach der Datenbank gerichtet werden.


\section{Weitere Werkzeuge}

Hier werden weitere Werkzeugarten und Technologien vorgestellt,
welche die Implementation unterst\"utzen werden.

\subsection{Testwerkzeuge}

Um die noch zu entwickelnden UnitTests,
sowie die bereits in Kapitel~\ref{chap:target} spezifizierten Tests umzusetzen,
is

\subsection{Datenbank Management}

\subsection{Technologische Betrachtung}
\begin{verbatim}
- tdd
- 
\end{verbatim}
\subsection{Produktanforderungen}
%-> in die implementation
\begin{verbatim}
- pytest ref
\end{verbatim}
