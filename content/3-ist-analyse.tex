\chapter{Ist-Analyse ( 5 \%)}
\label{chap:ist-analyse}


\section{Aufbau traditioneller Systeme}

\begin{verbatim}
- datensammlung
  - logs
    - in time
    - STDOUT,
  - files
    - junitxml
\end{verbatim}

\subsection{Vor- und Nachteile der existierenden Systeme}

\begin{verbatim}
- etablierung
- dokumentation
\end{verbatim}


\section{Auswahl Praxisbeisiele}

\section{Probleme Existierender Systeme}

\begin{verbatim}
- etablierung
- dokumentation
  - konfigurierbarkeit
\end{verbatim}

\subsection{Datenzugriff}
\begin{verbatim}
- eigene datenformate
\end{verbatim}
\subsection{Erweiterbarkeit}

\subsection{Komponentenabh\"angigkeit}
\begin{verbatim}
- zentrales management
- strikt master/slave
- alles wird im master gemanagt
- dumme arbeiter, keinerlei autonomie

- ausfall von master bedingt ausfall von slave


\end{verbatim}

\section{Praxisbeispiel Jenkins}

%XXX referenzen
Jenkins und Hudson stellen ein Benutzerfreundliches,
jedoch limitiertes System zum einfachen Anlegen von Build-Jobs.

Parametrisierung ist nicht möglich.

\section{Praxisbeispiel Buildbot}

%XXX referenzen
BuildBot positioniert sich als eine Art Meta-Build-Server.
es biete keine normale Oberfläche, sondern wird mittels
Komposition von Metadaten und Komponenten konfiguriert.

Die Konfiguration des Servers stellt dabei ein Python script,
welches die Operationen ausführt, welche zur Zusammenstellung des gewünschten Servers notwendig sind.

Es ist möglich Builds Jobs mit minimale Parameter in Form von Strings zu Übergeben.

\section{Praxisbeispiel TravisCI}

\begin{verbatim}
- matix builds
- branch builds
- mangel an datenzugriff
\end{verbatim}
