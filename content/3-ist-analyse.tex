\chapter{Ist-Analyse ( 5 \%)}

\section{Probleme Existierender Probleme}
\subsection{Datenzugriff}
\subsection{Erweiterbarkeit}
\subsection{Komponentenabh\"angigkeit}
\section{Auswahl Praxisbeisiele}
\section{Praxisbeispiel Jenkins}
\section{Praxisbeispiel Buildbot}
\section{Praxisbeispiel TravisCI}

\section{stuff}
\subsection{Überblick}

\subsubsection{buildbot}

%XXX referenzen
BuildBot positioniert sich als eine Art Meta-Build-Server.
es biete keine normale Oberfläche, sondern wird mittels
Komposition von Metadaten und Komponenten konfiguriert.

Die Konfiguration des Servers stellt dabei ein Python script,
welches die Operationen ausführt, welche zur Zusammenstellung des gewünschten Servers notwendig sind.

Es ist möglich Builds Jobs mit minimale Parameter in Form von Strings zu Übergeben.

\subsubsection{jenkins/hudson}

%XXX referenzen
Jenkins und Hudson stellen ein Benutzerfreundliches,
jedoch limitiertes System zum einfachen Anlegen von Build-Jobs.

Parametrisierung ist nicht möglich.

\subsection{Vor- und Nachteile der existierenden Systeme}

\begin{verbatim}
- buildbot/jenkins
  - konfigurierbarkeit
  - zentrales management
  - eigene datenformate

- etablierung
- dokumentation
\end{verbatim}


