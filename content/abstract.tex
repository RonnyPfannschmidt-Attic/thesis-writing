\begin{abstract}

Kontinuierliche Integration, kontinuierliches Testen und kontinuierliche Verteilung entwickeln sich in der modernen Softwareentwicklung immer mehr zu einem Unabdingbaren Werkzeug,
welches  maßgeblich bei der Früherkennung von Fehlern hilft.

Die Werkzeuge, die diese Verfahren ermöglichen, lassen jedoch einige Fragen und Probleme offen.
In dieser Arbeit werden diese Probleme zuerst erörtert. Anschließend werden Lösungswege auf Basis einer verteilten Datenbank vorgestellt
und als Prototyp implementiert.

Der Prototyp wird dann evaluiert und optimiert.


\end{abstract}
