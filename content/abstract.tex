\begin{abstract}
Diese Arbeit setzt sich zum Ziel experimentel und analytisch festzustellen,
wie moderne NoSql Datenbanken Einfluss auf die Entwicklung
von Datenbankgestützter Software nehmen.
Im Rahmen dieser Arbeit soll dabei der spezielle Fall der
Unterstützung/Konstruktion von Werkzeugen 
zur Kontinuierlichen Integrations betrachtet werden.

Besonderer Augenmerk liegt dabei auf dem Mehrwert welcher mit
den Eigenschaften einer NoSql Datenbank erzielt werden kann.

Ansatzpunkte für den Mehrwert sind dabei auf Applikationsebene 
insbesondere bessere Integration in den Arbeitsablauf von Entwicklern,
sowie interessante Werkzeuge für die Analyse von Resultaten.

Als Werkzeug und technischen Rahmen wird diese Arbeit das System CouchDB verwenden,
welches eine Schemalose DokumentDatenbank für Json Dokumente bereitstellt,
und mittels map/reduce erweiterte Möglichkeiten zur Datenanalyse bietet.

In diesem Kontext soll auch Untersucht werden,
wie die Eigenschaften der Datenbank Einfluss auf
die Entwicklung der Prototypischen Apllikation nehmen,
und welchen Einfluss dies auf die Erweiterbarkeit hat
(mit besonderem Fokus auf die Konzeptuelle Vereinfachung
von Erweiterungen für das System).

Zum Vergleich werden dabei Traditionalle Relationale Modellierung
sowie Existierende CI Tools herangezogen.


\end{abstract}
