\begin{abstract}
Diese Arbeit setzt sich zum Ziel experimentel und analytisch festzustellen,
wie moderne NoSql Datenbanken Einfluss auf die Entwicklung
von Datenbankgestützter Software nehmen.

Im Rahmen dieser Arbeit soll dabei der spezielle Fall der
Unterstützung/Konstruktion von Werkzeugen 
zur Kontinuierlichen Integration betrachtet werden.

Da diese Klasse von Werkzeugen aus modernen Arbeitabläufen,
welche Automatisiertes Testen und Verteilen beinhalten,
nicht mehr wegzudenken ist.


Besonderer Augenmerk liegt dabei auf dem Mehrwert welcher, mit
den Eigenschaften einer NoSql Datenbank, erzielt werden kann.

Ansatzpunkte für den Mehrwert sind dabei auf Applikationsebene 
insbesondere bessere Integration in den Arbeitsablauf von Entwicklern
und die Dahinterliegende Modellierung,
sowie dem Aufwand bei der Schaffung von Werkzeugen für die Analyse von Resultaten.

Als Datenbank und technischen Rahmen wird diese Arbeit das System CouchDB verwenden,
welches eine Schema-lose Dokument-Datenbank für JSON Dokumente bereitstellt,
und mittels map/reduce erweiterte Möglichkeiten zur Datenanalyse bietet.

In diesem Kontext soll auch Untersucht werden,
wie die Eigenschaften der Datenbank Einfluss auf
die Entwicklung der Prototypischen Applikation nehmen,
und welchen Einfluss dies auf die Erweiterbarkeit hat
(mit besonderem Fokus auf die Konzeptuelle Vereinfachung
von Erweiterungen für das System).

Zum Vergleich werden dabei Traditionelle Relationale Modellierung
sowie Existierende CI Tools herangezogen.

Endziel der Arbeit ist dabei die Aufstellung, Analyse und Zusammenfassung von
Praktischen und Theoretischen Beispielen welche den Einfluss von NoSql
auf die verschiedenen Ebenen der Applikations-Entwicklung aufzeigen

\end{abstract}
