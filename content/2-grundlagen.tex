\chapter{Grundlagen (5\%)}
\label{chap:base}
Dieses Kapitel beschreibt einige grundlegende Sachverhalte,
welche f\"ur das Verst\"andniss der Arbeit hilfreich sind.

\section{Verteilte Systeme}
\label{sec:base:vs}

\begin{verbatim}
- basics - copy somewhere
\end{verbatim}

\section{CAP}
\label{sec:base:cap}


%XXX literatur
% 
Das CAP Theorem beschreibt 3 Grundlegende

\begin{verbatim}
- basics - copy somewhere
\end{verbatim}

\section{BASE}
\label{sec:base:base}

\section{Kontinuierliche Integration/Verteilung}
\label{sec:base:ci}
\begin{verbatim}

- folwer ci - grundlegende verwendung
- http://jayflowers.com/joomla/index.php?option=com_content&task=view&id=26
\end{verbatim}

\section{Quelltext/Versions Management}
\label{sec:base:scm}


Quelltext und Versions-Management fasst eine Menge
von Praktiken und Werkzeugen zusammen,
welche dazu Dienen den Quellcode eines Projektes zu verwalten.
Sie verwalten \"Anderungen am Quelltext, Entwicklungszweige und sogennante Tags,
welche einen Namen zu einem bestimmten \"Anderungsstand zuweisen.

Dies wird vorwiegend genutzt um die Versionsnummer eines Releases an den dazugeh\"origen \"Anderungsstand zu binden.

%XXX


\begin{verbatim}
- git/hg?
- branches
- entwicklungsmodelle
\end{verbatim}
