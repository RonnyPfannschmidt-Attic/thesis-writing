\chapter{Grundlagen (5\%)}
\label{chap:base}
Dieses Kapitel beschreibt einige grundlegende Sachverhalte,
welche für das Verständnis dieser Arbeit hilfreich sind.

\section{Verteilte Systeme}
\label{sec:base:vs}

\begin{verbatim}
- basics
- cite from somewhere
\end{verbatim}

\section{CAP}
\label{sec:base:cap}
%XXX literatur
%  http://www.julianbrowne.com/article/viewer/brewers-cap-theorem
Das \ac{CAP} Theorem nach \cite{brewer:cap} beschreibt drei grundlegend
wünschenswerte Eigenschaften von verteilten Systemen/Datenbanken.

\begin{description}
  \item[Konsistenz (C)] \hfill \\
      Alle Knoten sehen zur selben Zeit die selben Daten. 
  \item[Verfügbarkeit (A)] \hfill \\
      Alle Anfragen an das System werden stets beantwortet.
  \item[Partitionstoleranz (P)] \hfill \\
      Das System arbeitet auch bei Verlust von Nachrichten,
      einzelner Netz-knoten oder bei der Partition des Netzes weiter.
\end{description}

Zu beachten ist dabei, dass nur zwei dieser drei Anforderungen gleichzeitig
vollständig erfüllt werden können.

\section{Kontinuierliche Integration/Verteilung - Traditionell}
\label{sec:base:ci}

Wie bereits im \cref{chap:intro} erwähnt, behandelt CI die regelmäßige Integration und das anschließende automatische Testen von Softwareprodukten
Dabei können verschiedenste Ereignisse den Anstoß geben.
AM einfachsten sind dabei der Anstoß durch eine Zeitsteuerung
oder vom Quellcode Management abgesetzte Mitteilungen über neue Änderungen.
Manche Systeme nehmen sogar den Erfolg anderer Tests zum Anstoß.

Ziel ist es dabei Fehler früher und schneller aufzufinden,
und zu verhindern, das diese erneut auftreten.
Außerdem wird ermöglicht, das Produkt immer in einem Zustand zu halten,
welcher das veröffentlichen einer neuen Version jederzeit ermöglicht.

Jay Folwer beschreibt in seinem Essay ``
A Recipe for Build Maintainability and Reusability'' \cite{folwer:receipe} eindrucksvoll in welchem Kontext diese Praxis einzusetzen ist..


\section{Quelltext/Versions-Management}
\label{sec:base:scm}


\ac{SCM} fasst eine Menge
von Praktiken und Werkzeugen zusammen,
welche dazu dienen, den Quellcode eines Projektes zu verwalten.
Sie verwalten Änderungen am Quelltext, Zweige der Entwicklung und sogenannte Tags,
welche einen Namen zu einem bestimmten Änderungsstand zuweisen.

Dies wird vorwiegend genutzt um die Versionsnummer einer neuen Version eines Software-Projektes an den dazugehörigen Änderungsstand zu binden.

Moderne Systeme Arbeiten dabei verteilt,
was dazu führt, das grundsätzlich immer mehrere Zweige der Entwicklung existieren.
Die bekanntesten Vertreter dieser Systeme sind \emph{Mercurial} \cite{mercurial:website}
und \emph{Git} \cite{git:website}.

