\chapter{Grundlagen (5\%)}
\label{chap:base}
Dieses Kapitel beschreibt einige grundlegende Sachverhalte,
welche f\"ur das Verst\"andniss der Arbeit hilfreich sind.

\section{Verteilte Systeme}
\label{sec:base:vs}

\begin{verbatim}
- basics - copy somewhere
\end{verbatim}

\section{CAP}
\label{sec:base:cap}


%XXX literatur
%  http://www.julianbrowne.com/article/viewer/brewers-cap-theorem
Das CAP Theorem nach \cite{brewer:cap} beschreibt 3 Grundlegend
w\"unschenswerte Eigenschaften von verteilten Systemen/Datenbanken.

\begin{description}
  \item[Konsistenz (C)] \hfill \\
      Alle Knoten sehen zur selben Zeit dieselben Daten. 
  \item[Verf\"ugbarkeit (A)] \hfill \\
      Alle Anfragen an das System werden stets beantwortet.
  \item[Partitionstoleranz (P)] \hfill \\
      Das System arbeitet auch bei Verlust von Nachrichten,
      einzelner Netzknoten oder Partition des Netzes weiter.
\end{description}

Zu beachten ist dabei, dass nur 2 dieser 3 Anforderungen gleichzeitig
vollst\"andig erfuellt werden k\"onnen.



%XXX



\section{Kontinuierliche Integration/Verteilung}
\label{sec:base:ci}
\begin{verbatim}


- nachgelagerte builds?
- folwer ci - grundlegende verwendung 
\end{verbatim}
\cite{folwer:receipe}

\section{Quelltext/Versions Management}
\label{sec:base:scm}


Quelltext und Versions-Management fasst eine Menge
von Praktiken und Werkzeugen zusammen,
welche dazu Dienen den Quellcode eines Projektes zu verwalten.
Sie verwalten \"Anderungen am Quelltext, Entwicklungszweige und sogenannte Tags,
welche einen Namen zu einem bestimmten \"Anderungsstand zuweisen.

Dies wird vorwiegend genutzt um die Versionsnummer eines Releases an den dazugeh\"origen \"Anderungsstand zu binden.

%XXX


\begin{verbatim}
- git/hg?
- branches
- entwicklungsmodelle
\end{verbatim}
