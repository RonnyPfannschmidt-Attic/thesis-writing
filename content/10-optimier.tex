\chapter{Optimierung}
\label{cha:opt}
Nachdem die Probleme festgestellt wurden,
müsste eigentlich ab \Cref{chap:ist-analyse} neu iteriert werden,
um die Anforderungen entsprechen anzupassen.
Um überhöhten Aufwand zu vermeiden,
sollen die Problemlösungen stattdessen hier vorgestellt werden.
Außerdem soll eine Auswahl auch implementiert werden.

\section{Langwierige Anlaufphase}
\label{cha:opt:anlaufphase}
\subsection{Analyse}

Das Problem der Anlaufphase lässt sich mit dem Verwenden der Mitteilungen über Änderungen erklären. Der Datenstrom beginnt dabei immer mit der ersten Änderung.
Somit steigt die Menge an Änderungen, welche nicht betrachtet werden, aber dennoch Teil der Mitteilungen sind, mit der Datenmenge in der Datenbank an.
Dies führt zwangsläufig dazu, dass die Primitiven \verb|watch_for| und \verb|listen_changes| mehr Zeit benötigen, um zur ersten verwertbaren Mitteilung vorzudringen.


\subsection{Lösungsansatz - Zustand Speichern}

Da es möglich ist, die Mitteilungen über Änderungen an einem bestimmten Änderungsstand der Datenbank beginnen zu lassen (update sequence),
kann durch Speichern und Wiederverwenden der aktuellen Sequenz verhindert werden,
dass das System auch die bereits fertiggestellten Änderungen betrachten muss.
Dazu werden neue Datenobjekte benötigt, welche die Zustände für die einzelnen Arbeiter festhalten. Eine Überarbeitung ab Grobkonzept wäre notwendig


\subsection{Lösungsansatz - CouchDB View}

Ein View in CouchDB ist in der Lage, Daten zu einem bestimmten Stand der Datenbank abzubilden und dieses Abbild zu speichern.
Damit ist es möglich, den aktuellen Zustand sowie den Stand der Änderungen zu erfragen und somit den Zustand des Views sowie die darauf-folgenden Änderungen zu bearbeiten 
Dies benötigt nur eine Überarbeitung der technischen Konzeption und der Implementation.

\subsection{Implementation}
Da der View \verb|stm| bereits die Anforderungen des View-Ansatzes erfüllt,
wurde er zur Implementation verwendet.

Die primitiven \verb|watch_for| und \verb|run_callbacks| wurden modifiziert.
Im neuen Ablauf bearbeiten sie zuerst die Daten aus dem View \verb|stm|.
Anschließend fahren sie wie bisher mit den nachfolgenden Änderungen fort.

\subsection{Resultat}
Das Problem ist damit eingestellt, das System hat keinerlei Wartezeiten mehr bei der Anlaufphase.


\section{Konfliktsituation Inanspruchnahme Arbeitspakete}
\label{cha:opt:conflict-claim}
\subsection{Analyse}
Die Konflikte bei der Inanspruchnahme von Arbeitspaketen ergeben sich aus der linearen Abarbeitung der Mitteilungen über Änderungen.
Jeder gerade freie Arbeiter hat die selbe Ansicht auf das nächste zur Verfügung stehende Arbeitspaket, deshalb fordern alle gleichzeitig dieses auch an.
Jedoch darf nur ein Arbeiter das Arbeitspaket auch bearbeiten,
alle anderen müssen es erneut versuchen.

\subsection{Lösungsansatz Token}
Der Lösungsansatz ``Token'' bedingt, dass die Auftragsvergabe auf das im Grobkonzept besprochene Meldeverfahren umgestellt wird.
Es gib dann nicht mehr die Möglichkeit, dass Arbeiter überhaupt Anspruch auf Arbeitspakete erheben, somit existiert der Konflikt auch nicht mehr.
Jedoch müssen dabei auch die Nachteile dieses Verfahrens in Kauf genommen werden.

\subsection{Lösungsansatz Zufall}

Der Lösungsansatz ``Zufall'' soll eine Zufallsfunktion dazu verwenden,
eines von mehreren verfügbaren Arbeitspaketen auszuwählen.
Durch die Steigerung der Menge kann dabei die Wahrscheinlichkeit des Konfliktes bei einzelnen Arbeitspaketen verringert werden. Dabei wird die Gleichverteilung eingesetzt, welche jedem Element in der Menge die gleiche Wahrscheinlichkeit gibt.

\subsection{Implementation}

Aufgrund der vorhergehenden Optimierung für die Anlaufphase
bietet die Primitive \verb|watch_for| einen geeigneten Ansatzpunkt für den Ansatz ``Zufall''.
Anstelle des nächsten Objektes im View werden die nächsten $N$ Objekte erfragt
und anschließend zufällig eines davon ausgewählt.
Dabei wurde in den Tests $N = 10$ gewählt.

\subsection{Resultat}
Im normalen Ablauf sind Konflikte nicht länger bemerkbar.
Sie sind nur noch feststellbar, wenn die Menge der verfügbaren Arbeitspakete
sehr gering ist oder bereits mehrere Arbeiter dazu übergegangen sind,
auf eine Mitteilung über einen neuen Auftrag zu warten.

Dies bedeutet, dass Konflikte nur noch dann auftreten,
wenn Arbeiter mangels weiterer verfügbarer Arbeitspakete nicht mehr in der Lage sind weiter zu arbeiten. Damit kann das Problem als gelöst betrachtet werden.

\section{Theoretische Betrachtung - Blockade Managementdaten}
\label{cha:opt:lock-management}
\subsection{Analyse}
Das Blockieren bei der Bearbeitung von Managementdaten lässt sich
aus der Kombination von zwei Umständen erklären.
Zum einen gibt die Primitive ``listen\_changes'' Mitteilungen
in der Reihenfolge der Datenbankoperationen zurück.
Zum anderen die Zusammenlegung von Zuteilung, Vorbereitung
und Erstellung von Arbeitspaketen.
Wird eine große Menge an Arbeitspaketen generiert,
so ist das Management anschließend lange Zeit damit beschäftigt,
diese für die Abarbeitung vorzubereiten.

In dieser Zeit kann der Manager aufgrund der Abarbeitungsreihenfolge
keine anderen Aufgaben wahrnehmen.
Somit werden andere Vorgänge blockiert.

\subsection{Lösungsansatz Prozess-Spaltung}

Der Ansatz der Spaltung teilt die Bearbeitung verschiedener Callbacks im Manager
in verschiedene Prozesse. Jeder Prozess bearbeitet dabei einen eigenen Strom an Mitteilungen, dies macht die verschiedenen Operationen unabhängig und neben-läufig.
Somit kann das Bearbeiten einer Art von Änderung nicht länger das Bearbeiten einer anderen beeinflussen.

Die Aufteilung in mehrere Prozesse benötigt eine Umstellung der technischen Konzeption und der Implementation.

\subsection{Lösungsansatz Priorisieren}

Bei der Priorisierung von Notifikationen ist das Ziel,
die Notifikationen, welche Einfluss auf andere Komponenten haben, zuerst zu betrachten.
Inanspruchnahme sollte also immer vor anderen Notifikationen bearbeitet werden.
Dies würde bedeuten, dass die Vor- und Nachbereitung nicht länger die Zuteilung von Arbeitspaketen behindern.

Um dies umzusetzen, wäre eine Neukonzeption im Grobkonzept notwendig.


\subsection{Ausgrenzung}

Aufgrund der relativen Komplexität der Umsetzung
wurde darauf verzichtet, die vorgestellten Lösungsansätze der Blockade im Management
als Teil dieser Arbeit zu testen.

\section{Blockade Datenausgabe Python}
\label{cha:opt:block-py-io}

\subsection{Analyse und Implementation}

Bei Recherchen im Internet stellte sich heraus,
dass Python in der Standardeinstellung die Datenausgabe puffert und somit Zeilen nicht direkt,
sondern in großen Blöcken ausgibt.
%XXX citation
Die Quellen legten nahe, den Interpreter einfach nur so zu konfigurieren,
dass er nach jeder Ausgabe auch den Puffer in den Datenstrom ausgibt.

\subsection{Resultat}

Die Ausgaben werden nun zeilenweise in das System gegeben.
Es kommt nicht mehr zu Blockaden.
