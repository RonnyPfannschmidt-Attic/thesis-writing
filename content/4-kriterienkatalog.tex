\chapter{Zielspezifikation ( 5 \%)}

%Was soll konkret gelöst werden. Hierzu gehören u.a.
%    -    Anforderungskatalog
%    -    Use-Cases
%    -    Pflichtenheft
%    -    Testkriterien
%    -    ... 


% kann/soll kriterien


\section{Anforderungen und Grundlegende Funktionen}

\begin{verbatim}
- annahme/generieren von aufrtraegen
- erstellung/management von arbeitspacketen
- abarbeitung von arbeitspacketen
- analyse von
  - zeitserien ueber projekte
  - auftraegen
  - arbeitspackete
\end{verbatim}

\section{Use Cases}

\begin{verbatim}
- alte usecases
  - junitxml beispiel
  - stdout beispiel

- neue use-cases
  - workdir diff/tarball
  - auswertung zeitserien <addon>
  - datenanalyse beispiel sommer <addon>
\end{verbatim}

\section{Pflichtenheft ?}

\begin{verbatim}
- programmteile
- siehe andforderungen/funktionen

\end{verbatim}

\section{Unit tests ?}


\begin{verbatim}
- verweis auf grobentwurf?

\end{verbatim}

\section{funktionale Tests}

\begin{verbatim}
- einzelnen komponenten
  - auftragseingang
  - aftragsvorbereitung
  - auftragsabarbeitung
  - schritte durchlaufen

- schrittypen durchfuehren
 - prozess
 - python ?
 - scm


\end{verbatim}

\section{systemtests}

\begin{verbatim}
- durchlauf komplettsystem
- resultate beispiel datenanalyse

\end{verbatim}

