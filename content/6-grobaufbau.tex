\chapter{Analyse \& Design ( 25 \%)}


\section{test}


\tikzstyle{decision} = [diamond, draw, fill=blue!20, 
    text width=4.5em, text badly centered, node distance=3cm, inner sep=0pt]
\tikzstyle{block} = [rectangle, draw, fill=blue!20, 
    text width=5em, text centered, rounded corners, minimum height=4em]
\tikzstyle{line} = [draw, -latex']
\tikzstyle{cloud} = [draw, ellipse,fill=red!20, node distance=3cm,
    minimum height=2em]

\begin{tikzpicture}[node distance = 2cm, auto]
    % Place nodes
    \node [block] (init) {initialize model};
    \node [cloud, left of=init] (expert) {expert};
    \node [cloud, right of=init] (system) {system};
    \node [block, below of=init] (identify) {identify candidate models};
    \node [block, below of=identify] (evaluate) {evaluate candidate models};
    \node [block, left of=evaluate, node distance=3cm] (update) {update model};
    \node [decision, below of=evaluate] (decide) {is best candidate better?};
    \node [block, below of=decide, node distance=3cm] (stop) {stop};
    % Draw edges
    \path [line] (init) -- (identify);
    \path [line] (identify) -- (evaluate);
    \path [line] (evaluate) -- (decide);
    \path [line] (decide) -| node [near start] {yes} (update);
    \path [line] (update) |- (identify);
    \path [line] (decide) -- node {no}(stop);
    \path [line,dashed] (expert) -- (init);
    \path [line,dashed] (system) -- (init);
    \path [line,dashed] (system) |- (evaluate);
\end{tikzpicture}






\section{\"Uberblick}
\subsection{Hauptkomponenten}
\subsection{Grundlegendes Datenmodell}

\section{Vorbereitende Primitiven}
\section{watch\_for}
\section{run\_callbacks}

\section{Projektkonfiguration}

\begin{verbatim}

- ingredient
  - vcs repos
  - links to tarballs
  - artifact?

- receipe
  - name
  - ingredients
  - declared simple build steps
  - known set of variation

- order
  - declares the execution of a certain receipe/set of receipes
  - declares a set of tasks to be executed
  - specific set of variations
  - status
- task
  - status
  - one specialisation of the declared specific set of variations of an order







- projekt
    - name
    - quellen
    - steps
    - default axen
    - abhaengigkeiten?
    - trigger
- integration
  - &projekt
  - grund
  - status
  - spez axen/axenwerte
- job
    - & projekt
    - & integration
    - spec (ausgewaehlte werte der axeen
    - status
    - worker


\end{verbatim}

\section{Auftragsannahme}
\subsection{Eingang}
\subsection{Validation}

\section{Management}
\subsection{Auftragsvorbereitung}
\subsection{Bereitstellung von Arbeitspacketen}
\subsection{Abschluss von auftr\"agen}


\section{Zuteilung/Abarbeitung von Arbeitspacketen}
\subsection{Zuteilungs}
\subsection{Vorbereitung Abarbeitung}
\subsection{Abschluss Abarbeitung}


\section{Abarbeitung von Arbeitspacketen}
\subsection{Arbeitsschritte}
\subsection{Datensammlung zur Laufzeit}
\subsection{Datensammling nach Abschluss eines Schrittes}
\subsection{Abschluss von Arbeitschritten}


\section{Arten von Arbeitschritten}
\subsection{Prozessaktionen}
\subsection{Quellcode Management Aktionen}
\subsection{Theoretische Betrachtung Interaktion}
