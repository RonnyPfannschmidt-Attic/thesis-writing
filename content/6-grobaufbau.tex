\chapter{Analyse \& Design ( 25 \%)}


\section{\"Uberblick}
\subsection{Hauptkomponenten}
\subsection{Grundlegendes Datenmodell}

\section{Vorbereitende Primitiven}
\section{watch\_for}
\section{run\_callbacks}

\section{Projektkonfiguration}

\begin{verbatim}

- ingredient
  - vcs repos
  - links to tarballs
  - artifact?

- receipe
  - name
  - ingredients
  - declared simple build steps
  - known set of variation

- order
  - declares the execution of a certain receipe/set of receipes
  - declares a set of tasks to be executed
  - specific set of variations
  - status
- task
  - status
  - one specialisation of the declared specific set of variations of an order







- projekt
    - name
    - quellen
    - steps
    - default axen
    - abhaengigkeiten?
    - trigger
- integration
  - &projekt
  - grund
  - status
  - spez axen/axenwerte
- job
    - & projekt
    - & integration
    - spec (ausgewaehlte werte der axeen
    - status
    - worker


\end{verbatim}

\section{Auftragsannahme}
\subsection{Eingang}
\subsection{Validation}

\section{Management}
\subsection{Auftragsvorbereitung}
\subsection{Bereitstellung von Arbeitspacketen}
\subsection{Abschluss von auftr\"agen}


\section{Zuteilung/Abarbeitung von Arbeitspacketen}
\subsection{Zuteilungs}
\subsection{Vorbereitung Abarbeitung}
\subsection{Abschluss Abarbeitung}


\section{Abarbeitung von Arbeitspacketen}
\subsection{Arbeitsschritte}
\subsection{Datensammlung zur Laufzeit}
\subsection{Datensammling nach Abschluss eines Schrittes}
\subsection{Abschluss von Arbeitschritten}


\section{Arten von Arbeitschritten}
\subsection{Prozessaktionen}
\subsection{Quellcode Management Aktionen}
\subsection{Theoretische Betrachtung Interaktion}
