\chapter{Implementation ( 15 \% ) }


\section{Werkzeuge}
\subsection{Datenbank}

Das Kriterium der \textbf{Master-Master Replikation}
schr\"ankte die Auswahl der verwendbaren Datenbanksysteme bereits stark ein.

In die n\"ahere Betrachtung kommen CouchDB (dokumentorientiert)
und PostgreSQL (relational).
Beide Systeme besizten Werkzeuge f\"ur Master-Master Replikation
und primitiven f\"ur Notifikationen.

%XXX: cites
Andere Systeme wie Mysql (relational), Mongodb(dokumentorientiert)
und  Neo4J (graph) scheiden aus, weil zu ihrem Funktionsumfang
nur die Master-Slave replikation geh\"ort.
Das System Riak wurde verworfen, da es \"Anderungs-notifikationen
nicht unterst\"utzte und in einem kurzen Vergleich
die Anfragen langsamer beantwortete.

Im Kurzvergleich von PostgreSQL mit CouchDB,
ging CouchDB als  hervor.
Im Test mussten bei Couchdb zwar Abstriche im Bereich Performance gemacht werden,
jedoch stellte sich heraus, das die \"Anderungsnotifikationen von Couchdb detailierter sind.
Ausserdem erscheint der Prozess der Replikation in CouchDB
als fehlertolleranter und einfacher.

% \cite
W\"ahrend in Postgresql die Notifikationen einfach nur eine Nachricht sind,
dass etwas ge\"andert wurde,
% \cite couchdb:changestream}
so beinhalten die Notifikationen von CouchDB, genaue Informationen
\"uber das ge\"anderte Objekt, eventuelle Datenbankkonflikte
und auf Wunsch sogar das Objekt selbst.

\subsection{Programmiersprachen}

Es wurden 2 Programmiersprachen Ausgew\"ahlt.
Zum einen Python, wegen seiner Bibliotheken und
zum anderen Javascript, da es die Datenbankinterne Sprache von Couchdb ist.

\begin{table}[h]
\centering
\begin{tabular}{l|c|c}
                            & \textbf{Python}   & \textbf{Javascript} \\
    \hline
    Datenbankintern         & nein              & ja \\
    Prozesskontrolle        & integriert        & extern, rudiment\"ar \\
    SCM Api                 & Verf\"ugbar (anyvc)& nicht verf\"ugbar \\
    CouchDB Client          & verg\"ugbar       & verf\"ugbar \\
\end{tabular}
\caption{\"Uberblick \"uber Features und Bibliotheken der Sprachen}
\label{tab:python-vs-js}
\end{table}

Ein besonderes Problem mit javascript ist,
dass es Datenbankintern externe Bibliotheken nicht
oder nur in limitierter Form verwenden kann.
Die Laufzeitumgebung weisst grosse Unterschiede
zwischen Datenbankinternen und Datenbankexterner verwendugn auf.

Somit ist Javascript f\"ur die Implementation der Komponenten nur sehr bedingt geeignet.

\subsection{Weitere Werkzeuge}

\subsubsection{Testwerkzeuge}
\subsubsection{Datenbank Management}

\section{Grundlegende Primitiven}

\subsection{listen\_changes}

\subsection{run\_callbacks}
\subsection{gather\_next}
\subsection{watch\_for}

\section{komponenten}

\subsection{Auftragsannahme}

\begin{verbatim}
- schritte
  - neu
  - bearbeiten
  - erhalten
  - validation
  - valid/invalid
  - bereitschaft
\end{verbatim}

\subsection{Auftragsverwaltung}

\begin{verbatim}
- erstellen von tasks
- erstellen von schritten
- bekanntmachung
\end{verbatim}

\subsection{auftragsvergabe}

\begin{verbatim}
- mvcc<> claim objekte
- anerkennung durch den master

\end{verbatim}

\subsection{Auftragsabarbeitung}
\begin{verbatim}
-lineares modell, ein schritt nach dem anderen
\end{verbatim}

\subsection{Prozessschritte}

\begin{verbatim}
- intro prozesse unix
- ausfuehrung
- datensammlung
- theoretische betrachtung stats
- \ldots
\end{verbatim}

\subsection{Quellcode Management Schritte}


\begin{verbatim}
- anyvc
- exemplarischer ablauf
    - checkout
    - update
\end{verbatim}

\section{Exemplarische Erweiterung - Datenaggregation}

 
%XXX: falscher platz
\begin{verbatim}

- beispiel sommer
  - ausgabe eines programms
  - zusammenfassung in graphen/uebersichten

\end{verbatim}

