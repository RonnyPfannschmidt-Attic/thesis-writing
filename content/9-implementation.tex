\chapter{Implementation}

Dieses Kapitel beschreibt die Implementation des Systems.
Dabei wird zuerst auf Wahl der Werkzeuge eingegangen,
anschließend wird die Struktur beschrieben.
Danach werden Primitiven vorgestellt, welche die Implementation der Komponenten vereinfachen.
Schließlich werden Besonderheiten bei der Implementation einiger Komponenten vorgestellt.
Zum Schluss werden die exemplarischen Erweiterungen vorgestellt.

\section{Werkzeuge}
\subsection{Datenbank}

Das Kriterium der \textbf{Master-Master Replikation}
schränkt die Auswahl der verwendbaren Datenbanksysteme bereits stark ein.

In die nähere Betrachtung kommen CouchDB \cite{couchdb:website} (dokumentorientiert)
und PostgreSQL \cite{postgresql:website} (relational).
Beide Systeme besitzen Werkzeuge für Master-Master-Replikation
und Primitiven für die Mitteilung von Änderungen.

Andere Systeme wie Mysql (relational) \cite{mysql:website}, Mongodb (dokumentorientiert) \cite{mongodb:website}
und  Neo4J (graph) \cite{neo4j:website} scheiden aus, weil zu ihrem Funktionsumfang
nur die Master-Slave-Replikation geh\"ort.
Das System Riak \cite{riak:website} wurde verworfen, da es Mitteilungen über Änderungen nicht unterstützte und in einem kurzen Vergleich
die Anfragen langsamer beantwortete.

Im oberflächlichen Vergleich von PostgreSQL mit CouchDB ging CouchDB als Sieger hervor.
Im Test mussten bei CouchDB zwar Abstriche im Bereich Performance gemacht werden,
jedoch stellte sich heraus, dass die Mitteilungen über Änderungen von CouchDB detaillierter sind.
Außerdem erscheint der Prozess der Replikation in CouchDB als fehlertoleranter und einfacher, da er fest integriert ist und für asynchrone Replikation mit langen Unterbrechungen konzipiert ist.

In PostgreSQL sind Mitteilungen ein extra Konzept \cite{postgresql:notify}.
Sie sind unabhängig von Änderungen der Datenbank und müssen manuell integriert werden. 
In CouchDB sind Mitteilungen \cite[Chap Notifications]{couchdb:guide}, genaue Informationen
über das geänderte Objekt, eventuelle Konflikte der Datenbank
und auf Wunsch sogar das Objekt selbst. Somit erfüllen die Mitteilungen in CouchDB bereits alle Anforderungen.

Somit wurde aufgrund der technischen Vorteile das System CouchDB als zugrunde liegende Datenbank ausgewählt.

\subsection{Programmiersprachen}

Es wurden zwei Programmiersprachen ausgew\"ahlt:
zum Einen Python wegen seiner Bibliotheken und
zum Anderen Javascript, da es die datenbankinterne Sprache von CouchDB ist.

\begin{table}[ht]
\centering
\begin{tabular}{l|c|c}
                            & \textbf{Python} & \textbf{Javascript} \\
    \hline
    Datenbankintern         & nein            & ja \\
    Prozesskontrolle        & integriert      & extern, rudimentär \\
    SCM API                & verfügbar       & nicht verfügbar \\
    CouchDB Client          & verfügbar       & verfügbar \\
\end{tabular}
\caption{Überblick über Features und Bibliotheken der Sprachen}
\label{tab:python-vs-js}
\end{table}

Ein besonderes Problem mit Javascript ist,
dass es innerhalb der Datenbank die Bibliotheken, welche Datenbank-extern sind,
nicht oder nur in limitierter Form verwenden kann.
Die Laufzeitumgebung weist große Unterschiede
zwischen Datenbank-interner und Datenbank-externer Verwendung auf.

Weiterhin konnten wie in \Cref{tab:python-vs-js} aufgezeigt,
für einige benötigte Bibliotheken keine
oder nur unzureichende Implementationen gefunden werden.
Somit ist Javascript für die Implementation der Komponenten
nur sehr bedingt geeignet.

Python hingegen hat Bibliotheken für alle notwendigen Grundfunktionen
und hat somit alle Voraussetzungen,
um die Implementation der Komponenten zu stützen.

\subsection{Bibliotheken}

Diese Unterkapitel stellt die genutzten Bibliotheken vor.

\subsubsection{Datenbank-Interaktion}

Zur Interaktion mit CouchDB wurde die Bibliothek ``\emph{couchdbkit}'' \cite{couchdbkit:website} verwendet.
Sie stellte als einzige alle Funktionen der HTTP-API von CouchDB zur Verfügung.
%XXX cites?
Andere Bibliotheken sind ``python-couchdb'' (kein Support für Änderungsmitteilungen)
und Paisley (sehr unvollständig).

\subsubsection{SCM Interaktion}

Für die Interaktion mit der Versions-Kontrolle wurde die Bibliothek ``anyvc'' \cite{anyvc:website} verwendet.
Sie ist eine von zwei Python Bibliotheken für den Zugriff auf SCM
und als einzige in der Lage mit einem Arbeitsverzeichnis zu interagieren.

\subsection{Weitere Werkzeuge}

\subsubsection{Testwerkzeuge}

Zum Testen wird das Test-Framework py.test \cite{pytest:website} verwendet.
Es das einzige Test-Framework für Python,
welches neben Werkzeugen für Unit-Tests auch Werkzeuge für
Funktionale-Tests und Akzeptanz-Tests mitbringt.

Mit der Erweiterung ``pytest-couchdbkit'' \cite{pytest:couchdbkit} ist es bereits
bestens für das Testen von Applikationen und einzelnen Funktionen,
welche CouchDB verwenden, geeignet.
Außerdem basiert ``pytest-couchdbkit'' auch auf der Bibliothek ``couchdbkit''
und ist damit bestens geeignet um die Basis für den Testprozess zu bilden.

\subsubsection{Datenbank-Management}

Zum Management der Datenbank an sich wurde die integrierte Administrationsschnittstelle Futon verwendet \cite{couchdb:futon}.
Sie bietet einfache Schnittstellen für die Verwaltung von Datenbanken sowie der Replikation.

Um Datenbankinterne Programme zu verwalten,
wurde das Werkzeug ``couchdb-compose'' \cite{couchdb:compose} geschaffen.
Die Umsetzung dieses Werkzeugs ist nicht als Teil dieser Arbeit zu betrachten.
Notwendig wurde es, da existierenden Werkzeugen wie ``Kanso'' und ``couchapp''
keine Möglichkeit boten, um direkte Kontrolle über den Aufbau und die Struktur 
einer CouchDB-Applikation zu üben.

\subsubsection{Javascript-Bibliotheken}

Um einige wiederkehrende Aufgaben in Javascript zu vereinfachen,
wurde die Bibliothek ``underscore'' \cite{javascript:underscore} verwendet.
Sie bietet diverse funktionale Werkzeuge, um das Schreiben
von Datenbank-internen Filtern und Views zu erleichtern.


\section{Projektstruktur}
\subsection{Überblick}

Die grundlegende Projektstruktur beinhaltet einige Hauptverzeichnisse.
Der Codename des Projektes ist dabei ``Juggler''.


\begin{description}
    \item[bin] beinhaltet die Skripts um einzelne Komponenten zu starten.
    \item[tool] beinhaltet Werkzeuge um mit der Datenbank des Prototyp zu interagieren.
    \item[juggler] beinhaltet die Implementation der Komponenten in Python.
    \item[composeapp] beinhaltet die couchdb-compose Applikation.
    \item[testing] beinhaltet die automatischen Tests.
    \item[example\_data] beinhaltet Beispiel-Datensätze im \ac{YAML} Format.
\end{description}

\subsection{Skripte und Werkzeuge}
Die Skripte und Werkzeuge haben verschiedene Aufgaben und Aufruf-formen,
diese werden hier beschrieben, dabei werden Parameter in eckigen Klammern angegeben.
Der Parameter namens \textbf{db} gibt dabei entweder einen kompletten Namen einer  Datenbank an oder den Namen einer lokalen Datenbank an.

\begin{description}
    \item[bin/slave.py] [db] [name] simple \hfill \\
        Started einen Slave/Worker Prozess.
    \item[bin/master.py] [db] \hfill \\
        Startet den Master Prozess - es darf nur einen geben und der Prototyp testet nicht, ob bereits einer aktiv ist.
    \item[tool/put.py] [db] [item] / [db] [item] --newid \hfill \\
        Überträgt den Inhalt der Datei namens \textit{item} zur Datenbank,
        ist das flag \textit{--newid} gesetzt, so wird eine neue Objekt-ID erzwungen.
    \item[tool/example-reset-db.sh] \hfill \\
        Shell-Skript, um die Datenbank für die manuellen Tests vorzubereiten.
        Es verwendet immer die lokale Datenbank namens ``juggler''.
    \item[tool/clean\_standing.py] [db]\hfill \\
        Setzt unterbrochene Arbeitspakete auf den Ursprungs-Zustand zurück.
        Dies ist notwendig, da der Prototyp den Status interrupted für ein Arbeitspaket nicht unterstützt.
\end{description}

\subsection{Python-Implementation}

Die Python-Implementation besteht aus einigen wichtigen Dateien im Ordner namens \verb|juggler|.

Einsprungpunkte sind die Dateien ``simple\_master.py'' und ``simple\_slave.py''.
Sie beinhalten die Implementationen der Management- und Arbeiter-Komponenten.

Hinzu kommt die Datei ``service.py'', welche das Grundgerüst für Komponenten enthält.

Folgende Verzeichnisse sind noch wichtig:

\begin{description}
    \dhitem[model] 
        enthält  die Modelklassen,
        welche die Daten aus der Datenbank auf Python Objekte .
    \dhitem[handler] 
        enthält die Einzelteile der Komponenten,
        welche in Kommunikation mit der Datenbank stehen.
    \dhitem[process]
        enthält die Implementation der Arbeitsschritte
        und ihrer Ausführungssumgebung.
\end{description}



\subsection{Datenbank Indexe}
\label{sec:db-indexe}

Die Indexe werden in CouchDB mittels 
eines MapReduce-Algorithmus \cite{couchdb:views} umgesetzt,
welcher die Resultate innerhalb der Datenbank speichert.

\begin{description}
    \dhitem[tasks\_of]
        Der \verb|tasks_of| View hat zwei Aufgaben.
        In der Map-Phase erzeugt er eine Menge an \hbox{$Key \rightarrow (Value(status), Id)$} Zuordnungen.
        Dabei ist der \textbf{Key} die ID des Auftrages
        \textbf{Id} ist die ID des Arbeitspaketes
        und \textbf{Value} der Zustand dieses Arbeitspaketes.
        Dies ermöglicht es, einfach herauszufinden,
        welche Arbeitspakete zu welchem Auftrag gehören.
        In der Reduce-Phase wird schließlich die eindeutige Menge
        der Zustände der Arbeitspakete gebildet.
        Besteht diese Menge ausschließlich aus finalen Zuständen,
        so kann der Auftrag als abgeschlossen angesehen werden.
    \dhitem[steps\_of]
        Der \verb|steps_of|-View hat lediglich die Aufgabe,
        eine Zuordnung von Arbeitsschritten
        zu ihren Arbeitspaketen zu schaffen.
    \dhitem[streams]
        Aufgabe dieses View ist es,
        festzustellen, welche Datenströme (stdout, stderr)
        eines Arbeitsschrittes Daten halten.
    \dhitem[lines]
        Der \verb|lines|-View dient dem Auffinden der einzelnen Zeilen
        eines Datenstroms. Sie werden nach der Zeilennummer geordnet.
    \dhitem[stm]
        Der \verb|stm| View hat mehrere Aufgaben.
        Er dient der statistischen Analyse und
        erzeugt seine Schlüssel aus der Kombination
        von Typ und Zustand eines Dokumentes.
        Dies ermöglicht es jederzeit festzustellen,
        welche Objekte sich in welchem Zustand befinden.
        In der Reduce-Phase wird die Anzahl ermittelt.
        Somit lässt sich jederzeit feststellen,
        welche Dokumente sich in welchem Zustand befinden.
%XXX: statret nach testing/eval
\end{description}

\subsection{CouchDB-Applikation}

Die Datenbank-Applikation wurde mit \emph{couchdb-compose} \cite{couchdb:compose} geschaffen und befindet sich wie bereits erwähnt im Verzeichnis \verb|composeapp|.

Die Grundstruktur ist dabei in der Datei \verb|couchdb-compose.yml| festgelegt.
Die bereits in \cref{sec:db-indexe} erwähnten Views befinden sich im Verzeichnis \verb|views|.
weiterhin beinhaltet die applikation einen \emph{list} \cite{couchdb:guide}[list] s names \verb|lines| - er dient dazu, die Daten aus dem \verb|lines| view als text zu formatieren.

Weiterhin sind in der Datei \verb|rewrites.yml| Regeln enthalten, um verwendbare URLs für den \verb|lines| show benutzen zu können.

\section{Grundlegende Primitiven}

Die grundlegenden Primitiven unterstützen die Implementation der einzelnen Teile des Prototypen. Die bilden die Basis für eine kompakte und schnell umsetzbare Implementation.

\subsection{listen\_changes}

Die Primitive \verb|listen_changes| dient dazu,
Änderungen der Datenbank den Komponenten der Applikation mitzuteilen.
Dabei kann nach dem Datentyp und nach weiteren Parametern gefiltert werden.
Somit kann man an eine Folge speziell ausgewählter Änderungen in der Datenbank erhalten.

\subsection{watch\_for}

Die Primitive \verb|watch_for| dient dazu, auf eine bestimmte Änderung in der Datenbank zu warten. Dazu setzt sie die \verb|listen_changes|-Primitive ein und gibt die erste zutreffende Änderung in der Datenbank zurück.

\subsection{watches\_for - Kontext für Komponenten}

Die Primitive \verb|watches_for| dient dazu, einzelne Systemkomponenten
an bestimmte Änderungen der Datenbank zu knüpfen.
Sie bringt die Zustandsänderungen zum Ausdruck,
bei denen eine Komponente aktiv werden soll.

Außerdem abstrahiert sie die Logik für das Warten auf diese Änderungen
(unter Verwendung der \verb|watch_for|-Primitive)
und ermöglicht es, die einzelnen Komponenten direkt mit den Daten aufzurufen,
auf die sie hätte warten müssen.
Dies ermöglicht es, einzelne Komponenten direkt zu testen, ohne dass eine echte Datenbank benötigt wird.

Die Primitive \verb|watches_for| umschließt dabei die Funktion,
welche die Arbeit verrichtet. Sie schafft somit durch Komposition eine neue Funktion,
welche basierend auf den übergebenen Parametern die Art und Weise der Vorarbeit entscheidet.

Wird das zu erwartende Objekt nicht mit übergeben, so werden aus den Parametern der Funktion die Parameter für die \verb|watches_for|-Primitive generiert.
Diese wird anschließend verwendet, um das zu erwartende Objekt von der Datenbank zu erhalten.



\subsection{run\_callbacks}

Die Primitive \verb|run_callbacks| dient dazu, mehrere Komponenten zu einem Programm
zu kombinieren, sie nimmt die Folge der Änderungen in der Datenbank und
ruft für jede Änderung die passende Komponente auf.

Einzelne Komponenten sind dabei mit der Primitive \verb|watches_for| vorbereitet,
die Primitive muss somit nur die Komponenten den Ereignissen zuordnen
und anschließend die Folge der Änderungen der Datenbank abarbeiten.

\section{Funktionale Komponenten}

Dieses Unterkapitel beschreibt die einzelnen Komponenten,
welche mit der \verb|watches_for|-Primitive kombiniert werden.
Sie bilden die Grundsteine des Prototypen.

In der ersten Komponente wird dabei die Verwendung der \verb|watches_for|-Primitive genauer vorgestellt und ein Code-Beispiel gegeben.

\subsection{Auftragsannahme}

Die Umsetzung der Auftragsannahme wurde stark vereinfacht.
Im Prototypen werden Aufträge mittels des \verb|put.py|-Skriptes in die Datenbank eingegeben.
Ein Beispiel ist in \Cref{fig:auftrag-beispieldaten}.

\begin{listing}[h]
\begin{minted}{yaml}
project: "print-some"
type: juggler:order
status: received
axis:
  first: [1,2,3]
  second: [3,4,5]
  third: [1,2,3]
  fourth: [3,4,5]
\end{minted}
\caption{Beispiel Auftr\"age im YAML Format}
\label{fig:auftrag-beispieldaten}
\end{listing}

Es bringt einen Auftrag zum Ausdruck, welcher vier Achsen der Konfiguration definiert.
Mit den Werten für diese Achsen werden somit $3*3*3*3 = 81$ Arbeitspakete benötigt.
Das zugehörige Projekt ist offensichtlich. Das Feld ``type'' gibt den für CouchDB benötigten Dokumenttyp an (es stellt nur eine Konvention dar).
Der Status \textit{``received''} sorgt dafür, dass dieser Auftrag sofort in Bearbeitung geht.

\FloatBarrier
Die Auftragsannahme im System beginnt dabei mit einem Auftrag der in den Zustand \textit{erhalten} (received) versetzt wurde.

Dieser sollte nun eigentlich validiert werden und anschließend in einen der Zustände valide oder invalide versetzt werden.

Da diese Umsetzung recht einfach ist, soll sie im \cref{fig:auftrag-validation-code} als Beispiel dienen.

\begin{listing}[h]
\mintedfromtexofcode{auftrag-validate}{juggler/handlers/inbox.py:order_validate}
\caption{Quelltext Ausschnitt Auftrag Validation}
\label{fig:auftrag-validation-code}
\end{listing}

Die Dekoration \cite{python:decorator}
der Funktion mit der \verb|watches_for|-Primitive legt den zu bearbeitenden Datentyp und Zustand fest.

Die Parameter der Funktion sind dabei zum Einen die Verbindung mit der Datenbank (db) und
zum Anderen das Objekt was erwartet oder übergeben wurde.
Der Ablauf ist nun denkbar einfach, der Status wird auf valide gesetzt und das Resultat wird wieder in der Datenbank gespeichert.
Damit ist der Eingang abgeschlossen.

\FloatBarrier
\subsection{Auftragsvorbereitung}

Bei der Arbeitsvorbereitung wird ein Auftrag,
welcher den Zustand \textit{valid} erreicht hat, um weiter Daten ergänzt.
Anschließend wird er in den Zustand \textit{ready} versetzt

\subsection{Erstellung Arbeitspakete}

Erreicht ein Auftrag den Zustand \textit{ready}, so ist er bereit für die Erstellung von Arbeitspaketen.
Bei der Erstellung der Arbeitspaketen soll für jede Konfiguration der Achsen ein Arbeitspaket generiert werden.
Dazu wird das kartesisches Produkt der Achsen gebildet.
Die Ergebnis-Menge beinhaltet dann alle Konfigurationen.
Die Menge aller Arbeitspakete muss nun in der Datenbank abgelegt werden
und der Auftrag erreicht seinen Endzustand.

Die generierten Arbeitspakete befinden sich dabei im Zustand \textit{new}.

\subsection{Generierung der Arbeitsschritte}

Ist ein neues Arbeitspaket vorhanden (Status \textit{new}), so müssen aus den Templates die eigentlichen Arbeitsschritte generiert werden.
Anschließend kann das Arbeitspaket als \textit{ready} (bereit) markiert werden.

\subsection{Inanspruchnahme von Arbeitspaketen}

Wie bereits in \Cref{sec:verfahren:erteilung} erwähnt,
gibt es 2 Möglichkeiten für die Umsetzung der Inanspruchnahme.
Da CouchDB \emph{MVCC} direkt unterstützt und das Verwenden von extra Objekten einiges an extra Aufwand generiert, wird die Methode der alternativen Versionen verwendet.
Die Inanspruchnahme eines Arbeitspaketes erzeugt also eine neue Version dieses Arbeitspaketes.
Diese beschreibt den neuen Besitzer (Feld ``owner'') und die Inanspruchnahme mittels des Zustandes \verb|claimed|
zum Ausdruck bringt.

\subsection{Zuteilung von Arbeitern}
Die Zuteilung von Arbeitern löst den Konflikt,
welchen die Inanspruchnahme in der Datenbank hinterlässt,
indem sie eine Version als Gewinner auswählt und
diese in den Zustand \textit{claimed} versetzt.

\subsection{Erwartung der Zuteilung}

Die ``Erwartung der Zuteilung'' dient dazu
einen Arbeiter in die Lage zu versetzen,
auf den Erfolg/Misserfolg einer Inanspruchnahme zu warten.
Dabei wird für einen bestimmten Auftrag darauf gewartet,
dass dieser den Zustand \textit{claimed} erreicht.

Erfolg/Misserfolg ergibt sich aus der Übereinstimmung zwischen dem Feld ``owner'' und dem aktuellen Arbeiter.

\subsection{Ausführung von Arbeitspaketen}

Die Ausführung von Arbeitspaketen dient dazu,
die Arbeitsschritte letztendlich innerhalb eines Arbeitsverzeichnisses abzuarbeiten.
Dieser Prozess wird später detaillierter beschrieben.


\section{Kombination der Komponenten}
\subsection{Inbox und Management}

Im Prototypen wurden Inbox und Management in einer einzelnen Systemkomponente  kombiniert.
Dabei kommt die ``run\_callbacks''-Primitive zum Einsatz.

Die Callbacks des Managers (und ihre Positionen im Quelltext) sind:
\begin{table}[h]
\begin{tabular}{lcl}
    \textbf{Callback} && \textbf{Position} \\
    Auftragsannahme && \verb|inbox.order_validate| \\
    Auftragsvorbereitung && \verb|inbox.valid_order_prepare| \\
    Generierung Arbeitspakete && \verb|inbox.ready_order_generate_tasks| \\
    Vorbereitung Arbeitspaket && \verb|manage.new_task_generate_steps| \\
    Zuteilung Arbeitspaket && \verb|manage.approve_claimed_task| \\
\end{tabular}
\caption{Callbacks des Managers und ihre Positionen im Quelltext}
\label{tab:callbacks-manager}
\end{table}

Die funktionalen Komponenten des Managers sind dabei so konzipiert,
dass ein Ausfall keine negativen Auswirkungen auf die Datenbank hat.
Er kann einfach an anderer Stelle neu gestartet werden.
Jedoch ist das System nicht in der Lage, korrekt zu agieren, wenn mehr als ein Manager aktiv ist.

\subsection{Arbeiter}

Arbeiter sind im Prototyp extrem einfach gebaut:
ihre Aufgabe besteht darin, sich Arbeitspakete zu sichern
und diese dann auszuführen.
Dazu nutzen sie in einer Schleife folgende Komponenten:

\begin{enumerate}
    \item Inanspruchnahme - \verb|work.claim_pending_task|
    \item Erwartung Zuteilung - \verb|work.wait_for_one_claiming_task|
    \item Ausführung - \verb|work.run_one_claimed_task|
\end{enumerate}

Wie sich aus der Reduktion auf ein Minimum schließen lässt,
wurde im Prototypen darauf verzichtet, den Abbruch oder die Unterbrechung
eines Arbeitspakets zu unterstützen.

Stattdessen dient das Hilfswerkzeug \verb|tool/clean_standing.py| dazu,
um Überbleibsel in der Datenbank nach dem erzwungenen Ende
eines Arbeiters zu entfernen.



\section{Procdir und Arbeitsschritte}

Procdir und Arbeitsschritte setzen die letztendliche Ausführung eines Arbeitspakets um.
Sie finden in der Ausführungskomponente im Arbeiter Anwendung.


\subsection{Procdir}

Der Lebenszyklus des Procdir entspricht der Ausführung des Arbeitspaketes.
Es verwaltet das Arbeitsverzeichniss des Arbeitspaketes und behandelt die Ausführung von Arbeitsschritten.
Es behandelt somit alle Vorgänge, welche von einem \emph{Arbeiter} Service benötigt werden, um  Arbeitspakete Schritt für Schritt auszuführen.

\subsection{Baseproc - eine Basis für Arbeitsschritte}

Die Baseproc Klasse stellt ein Framework für die Schaffung von eigenen Arbeitsschritten.
Neben der Verwaltung von zusammengehörenden Threads und Ressourcen bietet sie außerdem Werkzeuge, um Datensammlung zur Laufzeit zu behandeln.

Sie ermöglicht es, Arbeitsschritte als Kombination von Ressourcen und Threads (welche Laufzeit-Daten generieren) zu beschreiben.


\subsection{Prozessschritte}

Prozessschritte (\verb|process.subprocess.SubProcessProc|)
führen einzelne Prozesse aus und schließen ab, sobald dieser
abgeschlossen ist.

Dabei ist das Prozess-handle die grundlegende Ressource.
Auf seiner Basis sammeln mehrere Threads die Ausgaben des Prozesses sowie Informationen über Status der Ausführung des Prozesses.
Auf statistische Informationen wurde dabei verzichtet.

\subsection{Quellcode Management Schritte}

Die Quellcode Management Schritte sind auf Basis von anyvc \cite{anyvc:website} implementiert. Die wurden lediglich funktionsfähig erstellt und sammeln noch keine Laufzeitdaten.

\section{Exemplarische Erweiterung - Datenaggregation}

 
%XXX: falscher platz
\begin{verbatim}

- beispiel sommer
  - ausgabe eines programms
  - zusammenfassung in graphen/uebersichten

\end{verbatim}

