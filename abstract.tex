\begin{abstract}
Diese Arbeit setzt sich zum Ziel experimentel und analytisch festzustellen,
wie moderne NoSql Datenbanken zur Unterstützung/Konstruktion
Kontinuierlichen Integrations-Systemen genutzt werden können.

Besonderer Augenmerk liegt dabei auf dem Mehrwert welcher mit
den Eigenschaften einer NoSql Datenbank erzielt werden kann.

Ansatzpunkte für den Mehrwert sind dabei insbesondere
bessere Integration in den Arbeitsablauf von Entwicklern,
sowie interessante Werkzeuge für die Analyse von Resultaten.

Als Werkzeug und technischen Rahmen wird diese Arbeit
das System CouchDB verwenden,
welches eine Schemalose DokumentDatenbank für Json Dokumente bereitstellt,
und mittels map/reduce erweiterte Moeglichkeiten zur Datenanalyse bietet.

\end{abstract}
